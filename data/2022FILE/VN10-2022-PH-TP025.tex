
\setcounter{section}{0}
\begin{enumerate}[label=\bfseries Câu \arabic*:]
	\item \mkstar{2}
	
	{
		Vật lí có vai trò gì trong sản xuất các linh kiện điện tử, đặc biệt là chip điện tử?
	}
	
	\hideall{
		Được sử dụng phổ biến thế nhưng đã bao giờ bạn tự hỏi, thuật ngữ IC là gì, IC là viết tắt của từ gì chưa? Trong tiếng Anh, IC là viết tắt của cụm từ “integrated circuit”. IC còn được gọi là chip hay vi mạch tích hợp, vi mạch điện tử,… Đây là tập hợp của rất nhiều linh kiện bán dẫn và các linh kiện thụ động (như transistor và điện trở). Các linh kiện này được kết nối với nhau nhằm thực hiện một số chức năng xác định. IC được tạo ra để đảm nhiệm chức năng như một linh kiện kết hợp vô cùng quan trọng.
		
		Chip IC hay mạch tích hợp được cấu tạo từ vật liệu bán dẫn. Đây là một thiết bị điện tử có kích thước hình học rất nhỏ. Chip IC bao gồm một lượng lớn các linh kiện là transistor cũng như các linh kiện khác được chế tạo ở cùng một đế silic. 
		
		IC hay mạch tích hợp đóng vai trò vô cùng quan trọng trong đời sống, cụ thể ở lĩnh vực công nghệ thông tin. Hầu hết các thiết bị công nghệ hiện nay hầu hết đều sử dụng IC. Mạch tích hợp (IC) có khả năng làm giảm kích thước của mạch đi khá nhiều (cỡ vài micromet), bên cạnh đó chúng còn làm độ chính xác tăng lên. 
	}
	
	\item \mkstar{2}
	
	
	{
		Sóng dùng trong kĩ thuật rađa quân sự của quân chủng phòng không không quân khác gì với sóng được cá heo sử dụng để xác định phương hướng? Nguyên tắc vật lí nào được áp dụng để các máy bay tránh sóng rađa của đối phương?
	}
	
	\hideall
	{
		Áp dụng nguyên lí về phản xạ âm để chế tạo ra chiếc rađa hoàn chỉnh đầu tiên sử dụng trong quân sự vào năm 1935. Chỉ khác là rađa quân sự thường hoạt động ở tần số vô tuyến siêu cao tần, có bước sóng cực ngắn và có bản chất là sóng điện từ chứ không phải là sóng âm. 
		
		Kĩ thuật rađa phát triển khiến mọi ngóc ngách trên bầu trời không còn là bí mật đã thúc đẩy nhu cầu ngụy trang máy bay, nhất là máy bay trinh sát. Cơ sở cho việc ngụy trang trước rađa của đối phương là các lớp phủ bằng vật liệu có hệ số phản xạ thấp. Vật liệu này giúp tán xạ sóng từ rađa truyền tới theo nhiều hướng chứ không phản xạ trở lại rađa. Nguyên tắc này cũng tương tự như các bức tường trong nhà hát thường sần sùi và được phủ nhung để giảm tạp âm do phản xạ âm trên các bức tường.
		
	}
	\item \mkstar{2}
	
	
	{
		Các tế bào bình thường lớn lên, phân chia để hình thành tế bào mới thay thế tế bào cũ già dần và chết  đi. Những tế bào ung thư không chết đi mà liên tục phát triển và nhân lên không kiểm soát, tạo thành khối u. Đặc điểm nào của các tia phóng xạ đã được ứng dụng để điều trị ung thư?
	}
	
	\hideall
	{
		Với khả năng đâm xuyên mạnh, mang năng lượng cực lớn nên các tia phóng xạ đã được ứng dụng để điều trị ung thư
	}
	\item \mkstar{2}
	
	
	{
		Định luật vật lí nào đã được áp dụng để cải thiện công nghệ chế tạo động cơ máy bay?
	}
	
	\hideall
	{
		Khi những chiếc máy bay được cải tiến bằng những tiến bộ của vật lí, chúng sẽ có khả năng bay cao hơn và gặp những vấn đề mới nảy sinh. Càng lên cao, áp suất không khí càng giảm. Do đó, cần phải giữ áp suất trong máy bay ổn định ở điều kiện tương tự như tại mặt đất để đảm bảo sự tỉnh táo cho phi công khi vận hành máy bay. Sử dụng nguyên tắc tăng giảm áp suất trong vật lí, các nhà kĩ thuật đã chế tạo được khoang máy bay điều áp. Đây là một phát minh rất quan trọng trong ngành hàng không, đặc biệt là hàng không dân dụng bởi nó là yếu tố then chốt, giúp máy bay có thể bay cao hơn, lâu hơn và an toàn hơn.
	}
	\item \mkstar{2}
	
	
	{
		Lĩnh vực vật lí nào được áp dụng để truyền thông tin trong các máy điện báo?
	}
	
	\hideall
	{
		Với những tiến bộ trong lí thuyết về cảm ứng điện từ, các nhà vật lí đã có thể biến các rung động thành các tín hiệu điện, và ngược lại biến các tín hiệu điện thành rung động.
		
	}
	\item \mkstar{2}
	
	{
		Thảo luận và đánh giá những lợi ích, tác hại tiềm ẩn của công nghệ hạt nhân đối với nhân loại.
	}
	
	\hideall{
		
		\textbf{Lợi ích:}
		
		Thứ nhất, góp phần thúc đẩy tăng trưởng kinh tế và hạn chế tình trạng khai thác quá mức các nguồn tài nguyên gây tổn hại "các dịch vụ tự nhiên" như đa dạng sinh học, nguồn nước ngọt, không khí sạch và đất canh tác cũng như đe dọa sự phát triển bền vững. IAEA đã phát triển phương pháp mới, cho phép phân tích đồng thời và đồng bộ các tương tác phức tạp giữa thời tiết, sử dụng đất, các chiến lược năng lượng và nước..., nhằm tạo điều kiện thuận lợi giúp các nước dễ thích nghi hơn với các tình huống mới.
		
		Thứ hai, giúp xác định và xây dựng bản đồ các nguồn nước ngầm khả thi và nhanh hơn so với các công nghệ khác, từ đó nhân loại có thể dễ dàng tiếp cận các nguồn nước sạch và an toàn. Công nghệ hạt nhân cũng cải thiện hiệu quả các hệ thống thủy lợi hiện đang sử dụng tới $70\%$ nguồn nước ngọt của thế giới. 
		
		Thứ ba, giúp các nước sử dụng năng lượng hạt nhân an toàn và bền vững hơn, giảm thiểu lượng khí thải gây hiệu ứng nhà kính và tác động của biến đổi khí hậu. Trong bối cảnh nhu cầu điện của thế giới, năng lượng hạt nhân với sự giám sát trực tiếp của IAEA sẽ góp phần tăng cường hòa bình và an ninh trên thế giới. 
		
		Thứ tư, góp phần tăng sản lượng lương thực, đánh giá để bảo tồn và nâng cao độ phì nhiêu của các nguồn đất và quản lý nguồn nước trong bối cảnh an ninh lương thực vẫn là thách thức lớn của toàn cầu trong thập kỷ tới. 
		
		Thứ năm, giúp nhận thức và bảo vệ tốt hơn các đại dương của thế giới thông qua giám sát quá trình axít hóa trong các đại dương. Công nghệ hạt nhân cũng là công cụ để phát triển bức tranh tổng thể về các đại dương.
		
		Thứ sáu, cung cấp các chẩn đoán chính xác và quan trọng giúp phát hiện và điều trị các bệnh lây nhiễm và không lây nhiễm. Hàng triệu người trên thế giới hàng ngày đang phụ thuộc vào các phương pháp chẩn đoán và điều trị bệnh từ ứng dụng của công nghệ hạt nhân như dược phẩm phóng xạ... Sử dụng an toàn và phối hợp tốt công nghệ hạt nhân trong điều trị bệnh đang góp phần tích cực nâng cao sức khỏe và ổn định xã hội trên thế giới.
		
		\textbf{Tác hại:}
		
		Tạo ra chất thải nguy hại cho môi trường, quá trình vận hành có khả năng rủi ro và sự cố khá cao, chi phí xây dựng tốn kém cũng như tiềm tàng nguy cơ về an ninh hạt nhân.
	}
	
	\item \mkstar{2}
	
	
	{
		Vì sao phải nghiên cứu khí tượng, thủy văn? Vai trò của vật lí trong công tác dự báo thời tiết? Vật lí có vai trò gì trong xác định hải lưu, sóng biển, thủy triều?
		
	}
	
	\hideall
	{
		Khí tượng, thủy văn là ngành có vai trò quan trọng trong xu thế phát triển kinh tế xã hồi nhằm ứng phó với các vấn đề biến đổi khí hậu, thiên tại, lũ lụt, hạn hán,...
		
		Vật lí có vai trò quan trọng trong việc phân tích các yếu tố thời thiết để thiết lập các dự báo. Ngày nay, công tác dự báo thời tiết đã phát triển phương pháp và các thiết bị quan trắc, thu thập số liệu, phân tích, dự báo bằng phương pháp kĩ thuật số, radar và vệ tinh,...
		
		Sự phát triển của vệ tinh nhân tạo và công nghệ ảnh kĩ thuật số cho phép chụp ảnh bề mặt Trái Đất có độ nét cao, sử dụng ảnh viễn thám kết hợp với mô hình thủy văn để đánh giá những biến động của các đối tượng trên bề mặt Trái Đất để cung cấp cơ sở khoa học cho việc xác định các thông số của mô hình thủy văn, dự báo thời tiết và đánh giá sự biến đổi khí hậu.
	}
	\item \mkstar{2}
	
	
	{
		Những ứng dụng của vật lí trong nông nghiệp là gì và triển vọng cũng như tác động của vật lí đối với nông nghiệp thông minh như thế nào?
	}
	
	\hideall
	{
		Công nghệ nano được áp dụng để tăng hiệu quả và an toàn của phân bón và thuốc bảo vệ thực vật làm tăng sản lượng và chất lượng sản phẩm của cây lương thực, thời gian dự trữ rau quả, tạo tính chín sớm của cây trồng. 
		
		Công nghệ nhà kính được áp dụng phổ biến nhất trong nông nghiệp để tạo môi trường ổn định cho cây trồng và vật nuôi, chống côn trùng, bệnh tật lây lan.
		
		Chiếu xạ là phương pháp sử dụng bức xạ nhằm tiêu diệt các vi sinh vật còn tồn dư trong sản phẩm nông nghiệp, nhờ đó ngăn chặn sự lây lan của vi sinh vật và làm chậm hay loại bỏ mọc mầm hoặc chín, hỏng. 
		
		Tia hồng ngoại có tác dụng nhiệt rất mạnh, dùng để sấy khô các sản phẩm.
		
		Công nghệ hạt nhân sử dụng phương pháp chiếu xạ để tạo ra giống cây mới đạt năng suất cao.
		
		Công nghệ cơ khí và tự động hóa giúp chế tạo các máy nông nghiệp nhằm tiết kiệm sức lao động, thời gian, tăng năng suất lao động và chất lượng sản phẩm.
		
		
	}
	\item \mkstar{2}
	
	
	{
		Hãy tìm hiểu những ứng dụng của vật lí trong kinh tế và đánh giá triển vọng cũng như tác động của ngành này đối với khoa học và đời sống.
	}
	
	\hideall
	{
		Vật lí kinh tế là một lĩnh vực liên ngành, áp dụng các lí thuyết và phương pháp vật lí học để giải quyết các vấn đề trong kinh tế học.
	}
	\item \mkstar{2}
	
	
	{
		Hãy tìm hiểu những ứng dụng của vật lí trong lâm nghiệp và đánh giá triển vọng cũng như tác động của ngành này đối với khoa học và đời sống.
	}
	
	\hideall
	{
		Vật lí đóng góp quan trọng trong nghiên cứu nhân giống cây trồng, sản xuất và chế biến gỗ, nghiên cứu và phát triển các hệ thống quản lí và bảo vệ tài nguyên rừng, phòng chống cháy rừng.
		
		Hệ thống cảm biến giúp phân tích nhiệt độ, độ ẩm không khí, tốc độ gió, ... cung cấp dữ liệu cũng như cảnh báo đến điểm báo cháy tự động.
		
		Công nghệ viễn thám giúp theo dõi tài nguyên rừng được triển khai thuận lợi, chính xác cao, không tốn thời gian công sức nhiều.
		
		Công nghệ laser ứng dụng trong chết biến lâm sản có năng suất và tính an toàn cao. 
	}
\end{enumerate}