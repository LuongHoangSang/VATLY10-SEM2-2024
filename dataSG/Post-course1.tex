\setcounter{section}{0}
\section{Trắc nghiệm}
\begin{enumerate}[label=\bfseries Câu \arabic*:]
	\item \mkstar{1}
	
	
	{Khi nói về sự tương tác điện, trong các nhận định dưới đây, nhận định nào là \textbf{sa}i?
		\begin{mcq}
			\item Các điện tích cùng loại thì đẩy nhau.
			\item Các điện tích khác loại thì hút nhau.
			\item Hai thanh nhựa giống nhau, sau khi cọ xát với len dạ, nếu đưa lại gần thì chúng sẽ hút nhau.
			\item Hai thanh thủy tinh sau khi cọ xát vào lụa, nếu đưa lại gần nhau thì chúng sẽ đẩy nhau.
		\end{mcq}
	}
	
	\hideall
	{		\textbf{Đáp án: C.}
		
	Vì 2 thanh nhựa giống nhau khi cọ như nhau sẽ tích điện cùng loại và chúng sẽ phải đẩy nhau.
		
	}
		\item \mkstar{1}
	
	
	{Chọn phát biểu \textbf{sai}.
		\begin{mcq}
			\item Có hai loại điện tích là điện tích dương và điện tích âm.
			\item Các điện tích có thể hút nhau hoặc đẩy nhau.
			\item Hai quả cầu nhỏ nhiễm điện đặt xa nhau thì có thể coi chúng là các điện tích điểm.
			\item Khi hút nhau các điện tích sẽ dịch chuyển lại gần nhau.
		\end{mcq}
	}
	
	\hideall
	{		\textbf{Đáp án: D.}
		
		Lực tương tác tĩnh điện có độ lớn rất nhỏ nên không thể làm dịch chuyển các điện tích.
		
	}
		\item \mkstar{1}
	
	
	{Hai chất điểm mang điện tích $q_1$, $q_2$ khi đặt gần nhau chúng đẩy nhau. Kết luận nào sau đây \textbf{không} đúng?
		\begin{mcq}
			\item $q_1$ và $q_2$ đều là điện tích dương.
			\item $q_1$ và $q_2$ đều là điện tích âm.
			\item $q_1$ và $q_2$ trái dấu nhau.
			\item $q_1$ và $q_2$ cùng dấu nhau.
		\end{mcq}
	}
	
	\hideall
	{		\textbf{Đáp án: C.}
		
		Hai điện tích trái dấu thì không thể đẩy nhau.
		
	}
		\item \mkstar{1}
	
	
	{Điện tích điểm là
		\begin{mcq}
			\item vật có kích thước rất nhỏ.
			\item vật tích điện có kích thước rất nhỏ so với khoảng cách tới điểm mà ta xét.
			\item vật chứa rất ít điện tích.
			\item điểm phát ra điện tích.
		\end{mcq}
	}
	
	\hideall
	{		\textbf{Đáp án: B.}
		
		Điện tích điểm là vật tích điện có kích thước rất nhỏ so với khoảng cách tới điểm mà ta xét.
		
		
		
	}
		\item \mkstar{1}
	
	
	{Trong các cách sau, cách nào có thể làm nhiễm điện cho một vật?
		\begin{mcq}
			\item Cọ chiếc vỏ bút lên tóc.
			\item Đặt một nhanh nhựa gần một vật đã nhiễm điện.
			\item Đặt một vật gần nguồn điện.
			\item Cho một vật tiếp xúc với viên pin.
		\end{mcq}
	}
	
	\hideall
	{		\textbf{Đáp án: A.}
		
		Đây là hiện tượng nhiễm điện do co cọ xát.
		
	}
		\item \mkstar{1}
	
	
	{Nhận xét nào sau đây \textbf{không} đúng về điện môi?
		\begin{mcq}
			\item Điện môi là môi trường cách điện.
			\item Hằng số điện môi của chân không bằng 1.
			\item Hằng số điện môi của một môi trường cho biết lực tương tác giữa các điện tích trong môi trường đó nhỏ hơn so với khi chúng đặt trong chân không bao nhiêu lần.
			\item Hằng số điện môi có thể nhỏ hơn 1.
		\end{mcq}
	}
	
	\hideall
	{		\textbf{Đáp án: D.}
		
		Hằng số điện môi của chân không bằng 1 là nhỏ nhất.
		
	}
		\item \mkstar{1}
	
	
	{Độ lớn lực tương tác giữa hai điện tích điểm đặt trong không khí
		\begin{mcq}
			\item tỉ lệ với bình phương khoảng cách giữa 2 điện tích.
			\item tỉ lệ với khoảng cách giữa 2 điện tích.
			\item tỉ lệ nghịch với bình phương khoảng cách giữa 2 điện tích.
			\item tỉ lệ nghich với khoảng cách giữa 2 điện tích
		\end{mcq}
	}
	
	\hideall
	{		\textbf{Đáp án: C.}
		
		
		
	}
		\item \mkstar{1}
	
	
	{Khẳng định nào sau đây \textbf{không} đúng khi nói về lực tương tác giữa hai điện tích điểm trong chân không?
		\begin{mcq}
			\item có phương là đường thẳng nối hai điện tích.
			\item có độ lớn tỉ lệ với tích độ lớn hai điện tích.
			\item có độ lớn tỉ lệ nghịch với khoảng cách giữa hai điện tích.
			\item là lực hút khi hai điện tích trái dấu.
		\end{mcq}
	}
	
	\hideall
	{		\textbf{Đáp án: C.}
		
		Lực điện tỉ lệ nghịch với bình phương khoảng cách.
		
	}
		\item \mkstar{2}
	
	
	{Khi khoảng cách giữa hai điện tích điểm trong chân không giảm xuống 3 lần thì độ lớn lực Culông
		\begin{mcq}(4)
			\item tăng 9 lần.
			\item tăng 3 lần.
			\item giảm 9 lần.
			\item giảm 3 lần.
		\end{mcq}
	}
	
	\hideall
	{		\textbf{Đáp án: A.}
		
		
		$F$ tỉ lệ nghich với $r_2$, nên $r$ giảm 3 thì $F$ tăng 32 = 9 lần.
	}
		\item \mkstar{2}
	
	
	{Muốn lực tương tác giữa 2 điện tích điểm tăng 4 lần thì khoảng cách giữa chúng phải
		\begin{mcq}(4)
			\item tăng 4 lần.
			\item tăng 2 lần.
			\item giảm 4 lần.
			\item giảm 2 lần.
		\end{mcq}
	}
	
	\hideall
	{		\textbf{Đáp án: D.}
		
		$F$ tỉ lệ nghich với $r^2$, để $F$ tăng 4 thì $r$ giảm 2 lần.
		
	}
		\item \mkstar{2}
	
	
	{Tìm phát biểu \textbf{sai} về điện trường.
		\begin{mcq}
			\item Điện trường tồn tại xung quanh điện tích.
			\item Điện trường tác dụng lực điện lên các điện tích khác đặt trong nó.
			\item Điện trường của điện tích $Q$ ở các điểm càng xa $Q$ càng yếu.
			\item Xung quanh một hệ hai điện tích điểm đặt gần nhau chỉ có điện trường do một điện tích gây ra.
		\end{mcq}
	}
	
	\hideall
	{		\textbf{Đáp án: D.}
		
		
		
	}
	\item \mkstar{1}
	
	
	{Kết luận nào sau đây là \textbf{sai}?
		\begin{mcq}
			\item Đường sức điện trường là những đường có hướng.
			\item Đường sức điện đi ra từ điện tích dương và kết thúc là điện tích âm.
			\item Đường sức điện của điện trường tĩnh điện là đường khép kín.
			\item Qua mỗi điểm trong điện trường chỉ có một đường sức điện.
		\end{mcq}
	}
	
	\hideall
	{		\textbf{Đáp án: C.}
		
		
		
	}
	\item \mkstar{1}
	
	
	{Điện trường là
		\begin{mcq}
			\item môi trường không khí quanh điện tích.
			\item môi trường chứa các điện tích.
			\item môi trường bao quanh diện tích, gắn với điện tích và tác dụng lực điện lên các điện tích khác đặt trong nó.
			\item môi trường dẫn điện.
		\end{mcq}
	}
	
	\hideall
	{		\textbf{Đáp án: C.}
		
		
		
	}
	\item \mkstar{1}
	
	
	{Cường độ điện trường tại một điểm đặc trưng cho:
		\begin{mcq}
			\item thể tích vùng có điện trường là lớn hay nhỏ.
			\item điện trường tại điểm đó về phương diện dự trữ năng lượng.
			\item tác dụng lực của điện trường lên điện tích tại điểm đó.
			\item tốc độ dịch chuyển điện tích tại điểm đó.
		\end{mcq}
	}
	
	\hideall
	{		\textbf{Đáp án: C.}
		
		
		
	}
	\item \mkstar{1}
	
	
	{Vectơ cường độ điện trường tại mỗi điểm có chiều:
		\begin{mcq}
			\item cùng chiều với lực điện tác dụng lên điện tích thử dương tại điểm đó.
			\item cùng chiều với lực điện tác dụng lên điện tích thử tại điểm đó.
			\item phụ thuộc độ lớn điện tích thử.
			\item phụ thuộc nhiệt độ môi trường.
		\end{mcq}
	}
	
	\hideall
	{		\textbf{Đáp án: A.}
		
		
		
	}
	\item \mkstar{1}
	
	
	{Cường độ điện trường là đại lượng
		\begin{mcq}
			\item véctơ.
			\item vô hướng, có giá trị dương.
			\item vô hướng, có giá trị dương hoặc âm.       
			\item vectơ, có chiều luôn hướng vào điện tích.
		\end{mcq}
	}
	
	\hideall
	{		\textbf{Đáp án: A.}
		
		
		
	}
	\item \mkstar{1}
	
	
	{Trong các đơn vị sau, đơn vị của cường độ điện trường là
		\begin{mcq}(4)
			\item V/$m^2$.  
			\item V$\cdot$m. 
			\item V/m
			\item V$\cdot$m$^2$.
		\end{mcq}
	}
	
	\hideall
	{		\textbf{Đáp án: C.}
		
		
		
	}
	\item \mkstar{1}
	
	
	{Đặt một điện tích dương, khối lượng nhỏ vào một điện trường đều rồi thả nhẹ. Điện tích sẽ chuyển động.
		\begin{mcq}
			\item dọc theo chiều của đường sức điện trường.   
			\item ngược chiều đường sức điện trường.
			\item vuông góc với đường sức điện trường.  
			\item theo một quỹ đạo bất kỳ.
		\end{mcq}
	}
	
	\hideall
	{		\textbf{Đáp án: A.}
		
		
		
	}
	\item \mkstar{2}
	
	
	{Tại một điểm xác định trong điện trường tĩnh, nếu độ lớn điện tích thử tăng 2 lần thì độ lớn cường độ điện trường:
		\begin{mcq}(4)
			\item tăng 2 lần.    
			\item giảm 2 lần.  
			\item không đổi.   
			\item giảm 4 lần.
		\end{mcq}
	}
	
	\hideall
	{		\textbf{Đáp án: C.}
		
		Ta có cường độ điện không đổi với các điện tích thử khác nhau.
		
	}
	\item \mkstar{1}
	
	
	{Nếu khoảng cách từ điện tích nguồn đến điểm đang xét tăng 2 lần thì cường độ điện trường:
		\begin{mcq}(4)
			\item giảm 2 lần.  
			\item tăng 2 lần.   
			\item giảm 4 lần. 
			\item tăng 4 lần.
		\end{mcq}
	}
	
	\hideall
	{		\textbf{Đáp án: C.}
		
		
		
	}
\end{enumerate}


\hideall
{
	\begin{center}
		\textbf{BẢNG ĐÁP ÁN}
	\end{center}
	\begin{center}
		\begin{tabular}{|m{2.8em}|m{2.8em}|m{2.8em}|m{2.8em}|m{2.8em}|m{2.8em}|m{2.8em}|m{2.8em}|m{2.8em}|m{2.8em}|}
			\hline
			1.C  & 2.D  & 3.C  & 4.B  & 5.A  & 6.D  & 7.C  & 8.C  & 9.A  & 10.D  \\
			\hline
			11.D  & 12.C  & 13.C  & 14.C  & 15.A  & 16.A  & 17.C  & 18.A  & 19.C  & 20.C  \\
			\hline
			
		\end{tabular}
	\end{center}
}
\section{Tự luận}
\begin{enumerate}[label=\bfseries Câu \arabic*:]
	\item \mkstar{2}
	
	{
		Một electron có $q = -\text{1,6}\xsi{\cdot 10^{-19}}{C}$ và khối lượng của nó bằng $\text{9,1}\xsi{\cdot 10^{-31}}{kg}$. Xác định độ lớn gia tốc $a$ mà $e$ thu được khi đặt trong điện trường đều $E = \SI{100}{V/m}$.
	}
	
	\hideall{
		
		Ta có:
		
		$$F =|q|E = ma \Rightarrow a = \dfrac{|q|E}{m} = \text{1,785}\xsi{\cdot 10^{-3}}{m/s}^2.$$
		
	}
	
	\item \mkstar{2}
	
	{
	
	Tại một điểm có 2 cường độ điện trường thành phần vuông góc với nhau và có độ lớn là $\SI{3000}{V/m}$ và $\SI{4000}{V/m}$. Độ lớn cường độ điện trường tổng hợp là 
}
	
	\hideall{
	Vì $\vec{E_1}$ vuông góc $\vec{E_2}$ nên $E=\sqrt{E_1^2+E_2^2}=5000\ \text{V/m}$.
	}
	\item \mkstar{2}
	
	
	{
		Một quả cầu tích điện $+\SI{6.4e-7}{\coulomb}$. Trên quả cầu thừa hay thiếu bao nhiêu electron so với số proton để quả cầu trung hoà về điện?
		
	}
	
	\hideall
	{
		Vật mang điện tích dương $Q=+\SI{6.4e-7}{\coulomb}$, số electron thiếu là:
		
		$$n=\dfrac{q}{e}=\dfrac{\SI{6.4e-7}{\coulomb}}{\SI{1.6e-19}{\coulomb}}=4\cdot 10^{12}.$$
		
	}
	\item \mkstar{2}
	
	
	{
		Hai điện tích điểm giống nhau có độ lớn $\xsi{2\cdot 10^{-6}}{C}$, đặt trong chân không cách nhau $\SI{20}{cm}$ thì lực tương tác giữa chúng là bao nhiêu?
		
	}
	
	\hideall
	{
		Hai điện tích giống nhau nên cùng dấu, tương tác giữa hai điện tích là lực đẩy.
		
		$$F = k \dfrac{|q_1q_2|}{r^2} \Rightarrow F = \SI{0,9}{N}.$$
		
	}
	\item \mkstar{2}
	
	
	{
		
		Hai điện tích điểm $q_1=\text{2,5}\xsi{\cdot 10^{-6}}{C}$ và $q_2=\xsi{4\cdot 10^{-6}}{C}$ đặt gần nhau trong chân không thì lực đẩy giữa chúng là $\SI{1,44}{N}$. Khoảng cách giữ hai điện tích là bao nhiêu?
	}
	
	\hideall
	{
	Khoảng cách giữa hai điện tích 
	
	$$F = k \dfrac{|q_1q_2|}{r^2} \Rightarrow r = \sqrt{k \dfrac{|q_1q_2|}{F}} = \SI{0,25}{m} = \SI{25}{cm}.$$
	}
\end{enumerate}