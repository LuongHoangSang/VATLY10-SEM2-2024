\let\lesson\undefined
\newcommand{\lesson}{\phantomlesson{Ôn tập chương 6}}
%\hideall{
%	\inputansbox{10}{ans/VN10-2023-PH-TP0006-TN}
%}
\setcounter{ex}{0}
\Opensolutionfile{ans}[ans/VN10-2023-PH-TP0006-TN]
% ===================================================================
\begin{ex}\mkstar{1}
Chọn đáp án đúng nhất. Trường hợp nào sau đây có công cơ học?	
	\choice
	{Khi có lực tác dụng vào vật}
	{Khi có lực tác dụng vào vật và vật chuyển động theo phương vuông góc với lực}
	{Khi có lực tác dụng vào vật và vật đứng yên}
	{\True Khi có lực tác dụng vào vật và vật chuyển động}
	\loigiai{}
\end{ex}
% ===================================================================
\begin{ex}\mkstar{1}
	Công thức tính công cơ học khi lực $F$ làm vật dịch chuyển một quãng đường $s$ theo hướng của lực là	
	\choice
	{$A=\dfrac{F}{s}$}
	{\True $A=Fs$}
	{$A=\dfrac{s}{F}$}
	{$A=F-s$}
	\loigiai{Công thức tính công cơ học khi lực $F$ làm vật dịch chuyển một quãng đường $s$ theo hướng của lực là $A=Fs$.}
\end{ex}
% ===================================================================
\begin{ex}\mkstar{1}
	Trong các phát biểu sau, phát biểu nào đúng với định luật về công?
	\choice
	{Các máy cơ đơn giản đều cho ta lợi về công}
	{\True Không một máy cơ đơn giản nào cho ta lợi về công}
	{Không một máy cơ đơn giản nào cho ta lợi về lực}
	{Không một máy cơ đơn giản nào cho ta lợi về đường đi}
	\loigiai{Không một máy cơ đơn giản nào cho ta lợi về công. Được lợi bao nhiêu lần về lực thì thiệt bấy nhiêu lần về đường đi và ngược lại.}
\end{ex}
% ===================================================================
\begin{ex}\mkstar{1}
	Trong trường hợp nào dưới đây \textbf{không} có công cơ học?	
	\choice
	{Một người đang kéo một vật chuyển động}
	{\True Hòn bi đang chuyển động thẳng đều trên mặt sàn nằm ngang coi như tuyệt đối nhẵn}
	{Một lực sĩ đang nâng quả tạ từ thấp lên cao}
	{Máy xúc đất đang làm việc}
	\loigiai{Hòn bi đang chuyển động thẳng đều trên mặt sàn nằm ngang coi như tuyệt đối nhẵn không có công cơ học.}
\end{ex}
% ===================================================================
\begin{ex}\mkstar{1}
	Trong các phát biểu sau, phát biểu nào \textbf{sai}?	
	\choice
	{\True Ròng rọc cố định chỉ có tác dụng đổi hướng của lực và cho ta lợi về công}
	{Ròng rọc động cho ta lợi hai lần về lực, thiệt hai lần về đường đi}
	{Mặt phẳng nghiêng cho ta lợi về lực, thiệt về đường đi}
	{Đòn bẩy cho ta lợi về lực, thiệt về đường đi}
	\loigiai{Ròng rọc cố định chỉ có tác dụng đổi hướng của lực và không cho ta lợi về công.}
\end{ex}
% ===================================================================
\begin{ex}\mkstar{1}
	Công suất là
	\choice
	{công thực hiện được trong một giây}
	{công thực hiện được trong một ngày}
	{công thực hiện được trong một giờ}
	{\True công thực hiện được trong một đơn vị thời gian}
	\loigiai{}
\end{ex}
% ===================================================================
\begin{ex}\mkstar{1}
	Biểu thức tính công suất là	
	\choice
	{\True $\calP = \dfrac{A}{t}$}
	{$\calP = At$.}
	{$\calP= \dfrac{t}{A}$}
	{$\calP = \dfrac{A}{t^2}$}
	\loigiai{}
\end{ex}
% ===================================================================
\begin{ex}\mkstar{1}
	Đơn vị của công suất là
	\choice
	{$\si{\watt}$}
	{$\si{\kilo\watt}$}
	{$\si{\joule/\second}$}
	{\True Tất cả đều đúng}
	\loigiai{}
\end{ex}
% ===================================================================
\begin{ex}\mkstar{1}
	Vật có cơ năng khi
	\choice
	{\True vật có khả năng sinh công}
	{vật có khối lượng lớn}
	{vật có tính ì lớn}
	{vật đứng yên}
	\loigiai{Vật có cơ năng khi vật có khả năng sinh công.}
\end{ex}
% ===================================================================
\begin{ex}\mkstar{1}
	Thế năng hấp dẫn phụ thuộc vào những yếu tố nào?	
	\choice
	{Khối lượng}
	{Trọng lượng riêng}
	{\True Khối lượng và độ cao}
	{Khối lượng và vận tốc}
	\loigiai{}
\end{ex}
% ===================================================================
\begin{ex}\mkstar{1}
	Phát biểu nào sau đây đầy đủ nhất khi nói về sự chuyển hóa cơ năng?	
	\choice
	{Động năng có thể chuyển hóa thành thế năng}
	{Thế năng có thể chuyển hóa thành động năng}
	{Động năng và thế năng có thể chuyển hóa lẫn nhau, cơ năng không được bảo toàn}
	{\True Động năng và thế năng có thể chuyển hóa lẫn nhau, cơ năng được bảo toàn}
	\loigiai{}
\end{ex}
% ===================================================================
\begin{ex}\mkstar{1}
	Một người dùng một cần cẩu để nâng một thùng hàng có khối lượng 2500 kg lên độ cao 12 m từ mặt đất. Tính công thực hiện được trong trường hợp này.	
	\choice
	{\True $\SI{300}{\kilo\joule}$}
	{$\SI{250}{\kilo\joule}$}
	{$\SI{2.08}{\kilo\joule}$}
	{$\SI{300}{\joule}$}
	\loigiai{Công thực hiện được là
		$$A=10mh=\SI{300}{kJ}$$}
\end{ex}
% ===================================================================
\begin{ex}\mkstar{1}
	Một đầu máy xe lửa kéo các toa xe bằng lực $F=\SI{7500}{N}$. Công của lực kéo là bao nhiêu khi các toa xe chuyển động được quãng đường $s=\SI{8}{km}$?
	\choice
	{\True $\SI{60000}{kJ}$}
	{$\SI{60}{kJ}$}
	{$\SI{60000}{J}$}
	{Kết quả khác}
	\loigiai{Công thực hiện được là
		$$A=mgh=\SI{60000}{kJ}$$}
\end{ex}		
% ===================================================================
\begin{ex}\mkstar{1}
	Con ngựa kéo xe chuyển động đều với vận tốc $\SI{9}{km/h}$. Lực kéo là $\SI{200}{\newton}$. Công suất của ngựa nhận giá trị nào sau đây?
	\choice
	{$\SI{1500}{W}$}
	{\True $\SI{500}{W}$}
	{$\SI{1000}{W}$}
	{$\SI{250}{W}$}
	\loigiai{Công suất là
		$$\calP = \dfrac{A}{t} = Fv=\SI{500}{W}$$}
\end{ex}
% ===================================================================
\begin{ex}\mkstar{1}
	Một máy cơ trong 1 giờ sinh ra một công là $\SI{330}{kJ}$. Công suất của máy cơ này là
	\choice
	{$\SI{92.5}{W}$}
	{\True $\SI{91.7}{W}$}
	{$\SI{90.2}{W}$}
	{$\SI{97.5}{W}$}
	\loigiai{Công suất là
		$$\calP = \dfrac{A}{t} = \SI{91.7}{W}$$}
\end{ex}
% ===================================================================
\begin{ex}\mkstar{1}
	Người ta dùng một mặt phẳng nghiêng để kéo một vật. Nếu không có ma sát thì công cần thiết là $\SI{125}{\joule}$. Thực tế có ma sát nên công cần thiết là $\SI{175}{\joule}$. Hiệu suất của mặt phẳng nghiêng trên là bao nhiêu?
	\choice
	{$\SI{81.33}{\%}$}
	{$\SI{83.33}{\%}$}
	{\True $\SI{71.43}{\%}$}
	{$\SI{77.33}{\%}$}
	\loigiai{Hiệu suất:
		$$H=\dfrac{A_\text{ích}}{A_\text{toàn phần}} = \dfrac{A_F}{A_F + A_\text{ms}} =\SI{71.43}{\%} $$}
\end{ex}
% ===================================================================
\begin{ex}\mkstar{1}
	Một người đi xe đạp đều từ chân dốc đến đỉnh dốc cao 5 m. Dốc dài $\SI{40}{\meter}$, biết lực ma sát cản trở chuyển động của xe có độ lớn là $\SI{20}{\newton}$. Cả người và xe có khối lượng là $\SI{37.5}{kg}$. Công tổng cộng do người đó sinh ra là bao nhiêu?	
	\choice
	{$\SI{3800}{\joule}$}
	{$\SI{4200}{\joule}$}
	{$\SI{4000}{\joule}$}
	{\True $\SI{2675}{\joule}$}
	\loigiai{Công do người đó sinh ra:
		$$A=A_P + A_\text{ms} =\SI{2675}{\joule}.$$	}
\end{ex}
% ===================================================================
\begin{ex}\mkstar{1}
	Một cái máy bơm dùng để bơm nước vào ao. Một giờ nó bơm được $\SI{1000}{m^3}$ nước lên cao 2 m. Biết trọng lượng riêng của nước là $\SI{10000}{N/m^3}$. Công suất của máy bơm là	
	\choice
	{$\SI{5}{kW}$}
	{$\SI{5200.2}{W}$}
	{\True $\SI{5555.6}{W}$}
	{$\SI{5650}{W}$}
	\loigiai{Công suất máy bơm:
		$$\calP = \dfrac{A}{t} = \dfrac{Ph}{t} = \dfrac{dVh}{t} = \SI{5555.6}{W}.$$}
\end{ex}
% ===================================================================
\begin{ex}\mkstar{2}
	Trong quá trình chuyển động, nếu vật chỉ chịu tác dụng của trọng lực thì cơ năng của vật đó được tính bởi hệ thức	
	\choice
	{$W=2mgz+mv^2$}
	{$W=mgz+mv^2$}
	{$W=2mgz+\dfrac{1}{2}mv^2$}
	{\True $W=mgz+\dfrac{1}{2}mv^2$}
	\loigiai{Cơ năng của vật bằng tổng động năng và thế năng:
		$$W=mgz+\dfrac{1}{2}mv^2.$$}
\end{ex}
% ===================================================================
\begin{ex}\mkstar{2}
	Một vật có khối lượng $\SI{1}{kg}$ rơi tự do không vận tốc đầu từ độ cao $\SI{10}{m}$ xuống mặt đất. Lấy $g=\SI{10}{m/s^2}$. Cơ năng của vật bằng	
	\choice
	{$\SI{10}{J}$}
	{\True $\SI{100}{J}$}
	{$\SI{50}{J}$}
	{$\SI{5}{J}$}
	\loigiai{Cơ năng của vật bằng tổng động năng và thế năng (vì $v=0$ nên động năng bằng 0):
		$$W=W_\text t = mgh = \SI{100}{J}.$$}
\end{ex}
% ===================================================================
\begin{ex}\mkstar{2}
	Từ điểm M có độ cao $\SI{0.8}{m}$ so với mặt đất, người ta ném lên một vật có khối lượng $\SI{0.5}{kg}$ với vận tốc $\SI{2}{m/s}$. Lấy $g=\SI{10}{m/s^2}$. Cơ năng của vật bằng
	\choice
	{\True $\SI{5}{J}$}
	{$\SI{8}{J}$}
	{$\SI{4}{J}$}
	{$\SI{1}{J}$}
	\loigiai{	Cơ năng của vật bằng tổng động năng và thế năng:
		$$W=mgh + \dfrac{1}{2}mv^2 = \SI{5}{J}.$$}
\end{ex}
% ===================================================================
\begin{ex}\mkstar{2}
	Một vật nặng được thả rơi tự do từ độ cao $h$ so với mặt đất. Chọn gốc thế năng tại mặt đất. Bỏ qua mọi ma sát. Ngay trước khi vật chạm đất thì
	\choice
	{\True động năng cực đại, thế năng cực tiểu}
	{động năng bằng thế năng}
	{động năng cực tiểu, thế năng cực đại}
	{động năng bằng một nửa thế năng}
	\loigiai{Ngay trước khi vật chạm đất thì động năng cực đại vì $v$ cực đại, thế năng cực tiểu vì $h=0$ cực tiểu.}
\end{ex}
% ===================================================================
\begin{ex}\mkstar{2}
	Một vật có khối lượng $\SI{100}{g}$ được ném thẳng đứng lên cao với vận tốc $\SI{10}{m/s}$ từ độ cao $\SI{5}{m}$ so với mặt đất. Chọn gốc thế năng tại mặt đất. Lấy $g=\SI{10}{m/s^2}$. Cơ năng của vật khi chuyển động là	
	\choice
	{$\SI{15}{J}$}
	{$\SI{11.25}{J}$}
	{$\SI{10.5}{J}$}
	{\True $\SI{10}{J}$}
	\loigiai{Cơ năng của vật bằng tổng động năng và thế năng:
		$$W=mgh + \dfrac{1}{2}mv^2 = \SI{10}{J}.$$}
\end{ex}
% ===================================================================
\begin{ex}\mkstar{2}
	Một vật khối lượng $m=\SI{2}{kg}$ được ném theo phương thẳng đứng hướng xuống từ độ cao $\SI{15}{m}$ so với mặt đất với tốc độ $\SI{10}{m/s}$. Bỏ qua mọi lực cản. Chọn gốc thế năng tại mặt đất. Lấy $g=\SI{10}{m/s^2}$. Tốc độ của vật khi vật vừa chạm đất là
	\choice
	{$\SI{10}{m/s}$}
	{$\SI{15}{m/s}$}
	{\True $\SI{20}{m/s}$}
	{$\SI{400}{m/s}$}
	\loigiai{Bảo toàn cơ năng lúc vừa ném và lúc vừa chạm đất:
		$$W_1 = W_2 \Rightarrow W_\text{đ 1} + W_\text{t 1} = W_\text{đ 2} + W_\text{t 2} \Rightarrow \dfrac{1}{2}mv_1^2 + mgz_1 = \dfrac{1}{2}mv_2^2 + mgz_2.$$
		$$\Rightarrow \dfrac{1}{2}v_1^2 + gz_1 = \dfrac{1}{2}v_2^2 + gz_2 \Rightarrow v_2 = \SI{20}{m/s}.$$}
\end{ex}
% ===================================================================
\begin{ex}\mkstar{2}
	Một vật trượt trên mặt phẳng nghiêng có ma sát, sau khi lên tới điểm cao nhất nó trượt xuống vị trí ban đầu. Trong quá trình chuyển động trên	
	\choice
	{công của lực ma sát tác dụng vào vật bằng 0}
	{tổng công của trọng lực và lực ma sát tác dụng vào vật bằng 0}
	{\True công của trọng lực tác dụng vào vật bằng 0}
	{hiệu giữa công của trọng lực và lực ma sát tác dụng vào vật bằng 0}
	\loigiai{Một vật trượt trên mặt phẳng nghiêng có ma sát, sau khi lên tới điểm cao nhất nó trượt xuống vị trí ban đầu. Trong quá trình chuyển động trên công của trọng lực tác dụng vào vật bằng 0 vì công của trọng lực khi vật đi lên với khi vật đi xuống trái dấu, cùng độ lớn.}
\end{ex}
% ===================================================================
\begin{ex}\mkstar{2}
	Một thác nước cao $\SI{30}{\meter}$ đổ xuống phía dưới $\SI{10E4}{\kilogram}$ nước trong mỗi giây. Lấy $g =\SI{10}{\meter/\second^2}$, công suất thực hiện bởi thác nước bằng
	\choice
	{$\SI{2000}{\kilo\watt}$}
	{\True $\SI{3000}{\kilo\watt}$}
	{$\SI{4000}{\kilo\watt}$}
	{$\SI{5000}{\kilo\watt}$}
	\loigiai{Công xuất thác nước:
		$$\calP=\dfrac{mgh}{t}=\dfrac{\left(\SI{10E4}{\kilogram}\right)\cdot\left(\SI{10}{\meter/\second^2}\right)\cdot\left(\SI{30}{\meter}\right)}{\SI{1}{\second}}=\SI{3000}{\kilo\watt}.$$}
\end{ex}
% ===================================================================
\begin{ex}\mkstar{2}
	Một vật được thả rơi tự do không vận tốc đầu từ độ cao $h=\SI{60}{\meter}$ so với mặt đất. Chọn mốc tính thế năng tại mặt đất. Độ cao mà tại đó vật có động năng bằng ba lần thế năng là
	\choice
	{$\SI{20}{\meter}$}
	{\True $\SI{15}{\meter}$}
	{$\SI{10}{\meter}$}
	{$\SI{30}{\meter}$}
	\loigiai{$$W_\text{đ}=3W_\text{t}\Rightarrow W_\text{t}=\dfrac{1}{4}W\Rightarrow h=\dfrac{h_\text{max}}{4}=\SI{15}{\meter}.$$}
\end{ex}
% ===================================================================
\begin{ex}\mkstar{3}
	Một mũi tên khối lượng $\SI{75}{\gram}$ được bắn đi, lực trung bình của dây cung tác dụng vào đuôi mũi tên bằng $\SI{65}{\newton}$ trong suốt khoảng cách $\SI{0.9}{\meter}$. Mũi tên rời dây cung với tốc độ gần bằng
	\choice
	{$\SI{59}{\meter/\second}$}
	{\True $\SI{40}{\meter/\second}$}
	{$\SI{72}{\meter/\second}$}
	{$\SI{68}{\meter/\second}$}
	\loigiai{Áp dụng định lý biến thiên động năng cho mũi tên trong suốt quá trình chịu tác dụng lực bởi dây cung:
		$$W_\text{đ2}-W_\text{đ1}=Fs$$
		$$\Leftrightarrow \dfrac{1}{2}mv^2=Fs\Rightarrow v=\sqrt{\dfrac{2Fs}{m}}\approx\SI{39.5}{\meter/\second}.$$}
\end{ex}
% ===================================================================
\begin{ex}\mkstar{3}
	Một hòn đá có khối lượng $m =\SI{1}{\kilogram}$ ném thẳng đứng lên trên trong không khí với tốc độ ban đầu $v_0=\SI{20}{\meter/\second}$. Trong khi chuyển động vật luôn bị lực cản của không khí, coi lực cản có giá trị không đổi trong suốt quá trình chuyển động của hòn đá. Biết rằng hòn đá lên đến độ cao cực đại là $\SI{16}{\meter}$, lấy $g=\SI{9.8}{\meter/\second^2}$. Độ lớn của lực cản là
	\choice
	{$\SI{2.5}{\newton}$}
	{$\SI{5}{\newton}$}
	{\True $\SI{2.7}{\newton}$}
	{$\SI{0.25}{\newton}$}
	\loigiai{Công lực cản tác dụng lên vật bằng độ biến thiên cơ năng của vật:
		$$A_{F_c}=W_2-W_1$$
		$$\Leftrightarrow -F_c\cdot h=mgh-\dfrac{1}{2}mv^2_0$$
		$$\Rightarrow F_c=\dfrac{\dfrac{1}{2}mv^2_0-mgh}{h}=\SI{2.7}{\newton}.$$}
\end{ex}
% ===================================================================
\begin{ex}\mkstar{3}
	Vật đang chuyển động với tốc độ $\SI{25}{\meter/\second}$ thì trượt lên dốc. Biết dốc dài $\SI{50}{\meter}$, đỉnh dốc cao $\SI{14}{\meter}$, hệ số ma sát giữa vật và mặt dốc là $\mu_t=0,25$. Cho $g=\SI{10}{\meter/\second^2}$. Vận tốc ở đỉnh dốc là
	\choice
	{\True $\SI{10.25}{\meter/\second}$}
	{$\SI{33.80}{\meter/\second}$}
	{$\SI{25.20}{\meter/\second}$}
	{$\SI{9.75}{\meter/\second}$}
	\loigiai{Độ biến thiên cơ năng của vật nặng bằng công của lực ma sát:
		\begin{eqnarray*}
			&&A_{F_\text{ms}}=W_2-W_1\\
			&\Leftrightarrow& -F_\text{ms}\ell=mgh+\dfrac{1}{2}mv^2-\dfrac{1}{2}mv^2_0\\
			&\Leftrightarrow& -\mu_t mg\ell\cos\alpha=mgh+\dfrac{1}{2}mv^2-\dfrac{1}{2}mv^2_0\\
			&\Leftrightarrow&-0,25\cdot\left(\SI{10}{\meter/\second^2}\right)\cdot\left(\SI{50}{\meter}\right)\cdot\dfrac{\sqrt{\left(\SI{50}{\meter}\right)^2-\left(\SI{14}{\meter}\right)^2}}{\SI{50}{\meter}}=\left(\SI{10}{\meter/\second^2}\right)\cdot\left(\SI{14}{\meter}\right)+\dfrac{1}{2}v^2-\dfrac{1}{2}\cdot\left(\SI{25}{\meter/\second}\right)^2\\
			&\Rightarrow& v\approx\SI{10.25}{\meter/\second}
	\end{eqnarray*}}
\end{ex}
\Closesolutionfile{ans}