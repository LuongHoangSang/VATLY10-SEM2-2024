\setcounter{section}{0}

\begin{enumerate}[label=\bfseries Câu \arabic*:]
	\item \mkstar{1}
	
	
	{
		Đơn vị của động lượng bằng
		\begin{mcq}(4)
			\item N/s.
			\item N$\cdot$s. 
			\item N$\cdot$m.
			\item N$\cdot$m/s.
		\end{mcq}
	}
	
	\hideall
	{	
		\textbf{Đáp án: B.}
	}
	\item \mkstar{1}
	
	
	{Điều nào sau đây \textbf{sai} khi nói về động lượng?
		\begin{mcq}
			\item Động lượng của một vật có độ lớn bằng tích khối lượng và tốc độ của vật.
			\item Trong hệ kín, động lượng của hệ được bảo toàn.
			\item Động lượng của một vật có độ lớn bằng tích khối lượng và bình phương vận tốc.
			\item Động lượng của một vật là một đại lượng véc tơ.
		\end{mcq}
	}
	
	\hideall
	{	
		\textbf{Đáp án: B}
	}
	\item \mkstar{1}
	
	
	{Chọn câu phát biểu đúng nhất?
		\begin{mcq}
			\item Véctơ động lượng của hệ được bảo toàn. 
			\item Véctơ động lượng toàn phần của hệ được bảo toàn. 
			\item Véctơ động lượng toàn phần của hệ kín được bảo toàn. 
			\item Động lượng của hệ kín được bảo toàn.
		\end{mcq}
	}
	
	\hideall
	{	
		\textbf{Đáp án: C.}
	}
	\item \mkstar{1}
	
	
	{Động lượng của vật bảo toàn trong trường hợp nào sau đây? 
		\begin{mcq}
			\item Vật đang chuyển động thẳng đều trên mặt phẳng nằm ngang.
			\item  Vật đang chuyển động tròn đều. 
			\item Vật đang chuyển động nhanh dần đều trên mặt phẳng nằm ngang không ma sát. 
			\item Vật đang chuyển động chậm dần đều trên mặt phẳng nằm ngang không ma sát.
		\end{mcq}
	}
	
	\hideall
	{	
		\textbf{Đáp án: A.}
	}
	\item \mkstar{1}
	
	
	{Sở dĩ khi bắn súng trường các chiến sĩ phải tì vai vào báng súng vì hiện tượng giật lùi của súng có thể gây chấn thương cho vai. Hiện tượng súng giật lùi trên trên liên quan đến 
		\begin{mcq}(2)
			\item chuyển động theo quán tính.  
			\item chuyển động do va chạm.
			\item chuyển động ném ngang.
			\item chuyển động bằng phản lực.
		\end{mcq}
	}
	
	\hideall
	{	
		\textbf{Đáp án: D.}
	}
	\item \mkstar{2}
	
	
	{	Quả cầu A khối lượng $m_1$ chuyển động với vận tốc $\vec v_1$ va chạm vào quả cầu B khối lượng $m_2$ đứng yên. Sau va chạm, cả hai quả cầu có cùng vận tốc $\vec v_2$. Ta có hệ thức
		\begin{mcq}(2)
			\item $m_1\vec v_1 = (m_1 + m_2) \vec v_2.$
			\item  $m_1\vec v_1 = -m_2 \vec v_2.$
			\item $m_1\vec v_1 = m_2 \vec v_2.$
			\item $m_1\vec v_1 = \dfrac{1}{2}(m_1 + m_2) \vec v_2.$
		\end{mcq}
	}
	
	\hideall
	{	
		\textbf{Đáp án: A.}
	}
	\item \mkstar{2}
	
	
	{Một vật có khối lượng $\SI{4}{kg}$ rơi tự do không vận tốc đầu trong khoảng thời gian $\SI{2,5}{s}$. Lấy $g = \SI{10}{m/s}^2$. Độ biến thiên động lượng của vật trong khoảng thời gian đó có độ lớn là
		\begin{mcq}(4)
			\item $\SI{100}{kg\cdot m/s}$.
			\item $\SI{25}{kg\cdot m/s}$.
			\item $\SI{50}{kg\cdot m/s}$.
			\item $\SI{200}{kg\cdot m/s}$.
		\end{mcq}
	}
	
	\hideall
	{	
		\textbf{Đáp án: A.}
		
		Vận tốc ban đầu của vật: 
		
		$$v_0 = \SI{0}{m/s}.$$
		
		Vận tốc của ngay trước khi chạm đất
		
		$$v =gt = \SI{25}{m/s}.$$
		
		Độ biến thiên động lượng của vật trong khoảng thời gian:
		
		$$\Delta p = p_2 - p_1 = mv -mv_0 = \SI{100}{kg \cdot m/s}.$$
		
	}
	\item \mkstar{2}
	
	
	{Người ta ném một quả bóng khối lượng $\SI{500}{g}$ cho nó chuyển động với vận tốc $\SI{20}{m/s}$. Xung lượng của lực tác dụng lên quả bóng là
		\begin{mcq}(4)
			\item $\SI{10}{N\cdot s}.$
			\item $\SI{200}{N\cdot s}.$
			\item $\SI{100}{N\cdot s}.$
			\item $\SI{20}{N\cdot s}.$
		\end{mcq}
	}
	
	\hideall
	{	
		\textbf{Đáp án: A.}
		
		Xung lượng của lực tác dụng là
		
		$$F \cdot \Delta = \Delta p = mv = \SI{10}{N\cdot s}.$$
	}
	\item \mkstar{2}
	
	
	{Hai vật có khối lượng $m_1 = 2m_2$, chuyển động với vận tốc có độ lớn $v_1 = 2v_2$. Động lượng của hai vật có quan hệ
		\begin{mcq}(4)
			\item $p_1 = 2p_2.$
			\item $p_1 = 4p_2.$
			\item $p_2 = 4p_1.$
			\item $p_1 = p_2.$
		\end{mcq}
	}
	
	\hideall
	{	
		\textbf{Đáp án: B.}
		
		Ta có:
		
		$$\dfrac{p_1}{p_2} = \dfrac{m_1v_1}{m_2v_2} = \dfrac{2 \cdot 2 m_2v_2}{m_2v_2} = 4.$$
		
		Suy ra: $p_1 = 4p_2.$
	}
	\item \mkstar{2}
	
	
	{Một chất điểm chuyển động không vận tốc đầu dưới tác dụng của lực $F = \xsi{10^{-2}}{N}$. Động lượng chất điểm ở thời điểm $t = \SI{3}{s}$ kể từ lúc bắt đầu chuyển động là 
		\begin{mcq}(4)
			\item $\xsi{2 \cdot 10^{-2}}{kg \cdot m/s}.$
			\item $\xsi{3 \cdot 10^{-2}}{kg \cdot m/s}.$
			\item $\xsi{ 10^{-2}}{kg \cdot m/s}.$
			\item $\xsi{6 \cdot 10^{-2}}{kg \cdot m/s}.$
		\end{mcq}
	}
	
	\hideall
	{	
		\textbf{Đáp án: B.}
		
		Ta có:
		
		$$F = ma = m \dfrac{\Delta v}{\Delta t} = \dfrac{\Delta p}{\Delta t} = \dfrac{p}{t} \Rightarrow p = Ft = \xsi{3\cdot 10^{-2}}{kg\cdot m/s}.$$
	}
	\item \mkstar{2}
	
	
	{Một chiếc xe khối lượng $\SI{10}{kg}$ đang đỗ trên mặt sàn phẳng nhẵn. Tác dụng lên xe một lực đẩy $\SI{80}{N}$ trong khoảng thời gian $\SI{2}{s}$, thì độ biến thiên vận tốc của xe trong khoảng thời gian này có độ lớn bằng 
		\begin{mcq}(4)
			\item $\SI{1,6}{m/s}.$
			\item $\SI{0,16}{m/s}.$ 
			\item $\SI{16}{m/s}.$
			\item $\SI{160}{m/s}.$
		\end{mcq}
	}
	
	\hideall
	{	
		\textbf{Đáp án: C.}
		
		Ta có:
		
		$$\Delta p = p_2 - p_1 = m(v_2-v_1) = mv_2. (v_1 = 0).$$
		
		Lại có:
		
		$$\Delta p = F\Delta t \Rightarrow mv_2 = F\Delta t \Rightarrow v_2 = \SI{16}{m/s}.$$
	}
	\item \mkstar{2}
	
	
	{Chiếc xe chạy trên đường ngang với vận tốc $\SI{10}{m/s}$ va chạm mềm vào một chiếc xe khác đang đứng yên và có cùng khối lượng. Biết va chạm là va chạm mềm, sau va chạm vận tốc hai xe là
		\begin{mcq}(2)
			\item $v_1 = 0; v_2 = \SI{10}{m/s}.$ 
			\item $v_1 = v_2 = \SI{5}{m/s}.$ 
			\item $v_1 = v_2 = \SI{10}{m/s}.$
			\item $v_1 = v_2 = \SI{20}{m/s}.$
		\end{mcq}
	}
	
	\hideall
	{	
		Áp dụng định luật bảo toàn động lượng cho lúc trước và sau va chạm ta có:
		
		$$\vec p_1 = \vec p_2 \Leftrightarrow m\vec v = 2m \vec v' \Leftrightarrow v = 2v' \Rightarrow v' = \dfrac{v}{2} = \SI{5}{m/s}.$$
		
		Suy ra:
		
		$$v_1 = v_2 = \SI{5}{m/s}.$$
	}
	\item \mkstar{2}
	
	
	{Một đầu đạn khối lượng $\SI{10}{g}$ được bắn ra khỏi nòng của một khẩu súng khối lượng $\SI{5}{kg}$ với vận tốc $\SI{600}{m/s}$. Nếu bỏ qua khối lượng của vỏ đạn thì vận tốc giật của súng là
		\begin{mcq}(4)
			\item $\SI{12}{cm/s}$. 
			\item $\SI{1,2}{m/s}.$ 
			\item $\SI{12}{m/s}.$
			\item $\SI{1,2}{cm/s}.$ 
		\end{mcq}
	}
	
	\hideall
	{
	\textbf{Đáp án: B.}
			
	Trước khi bắn: $p_0 = 0$.
	
	Sau khi bắn:
	
	$$m_\text{s}v_\text{s} + m_\text{đ}v_\text{đ} = 0.$$
	
	Suy ra $v_\text{s} = - \dfrac{m_\text{đ}v_\text{đ}}{m_\text{s}} = -\SI{1,2}{m/s}.$
	
	}
	\item \mkstar{2}
	
	
	{Khối lượng súng là $\SI{4}{kg}$ và của đạn là $\SI{50}{g}$. Lúc thoát khỏi nòng súng, đạn có vận tốc $\SI{800}{m/s}$. Vận tốc giật lùi của súng là
		\begin{mcq}(4)
			\item $\SI{6}{m/s}$. 
			\item $\SI{7}{m/s}$. 
			\item $\SI{10}{m/s}$. 
			\item $\SI{12}{m/s}$. 
		\end{mcq}
	}
	
	\hideall
	{	\textbf{Đáp án: C.}
		
		Chọn chiều dương là chiều chuyển động của viên đạn sau khi bắn: $p_\text{t} = 0.$
		
		$$p_\text{s} = m_2v_2 - m_1v_1 = 40 - 4v_1.$$
		
		Áp dụng định luật bảo toàn động lượng:
		
		$$p_\text{t} = p_\text{s} \Leftrightarrow 0 = 40 - 4v_1 \Rightarrow v_1 = \SI{10}{m/s}.$$ 
	}
	\item \mkstar{2}
	
	
	{Một hòn bi khối lượng $m$ đang chuyển động với vận tốc $v$ đến va chạm mềm vào hòn bi thứ 2 khối lượng $3m$ đang nằm yên. Vận tốc hai viên bi sau va chạm là
		\begin{mcq}(4)
			\item $\dfrac{v}{3}.$
			\item $\dfrac{v}{4}.$ 
			\item $\dfrac{3v}{5}.$
			\item $\dfrac{v}{2}.$
		\end{mcq}
	}
	
	\hideall
	{	
		\textbf{Đáp án: B.}
		
		Hệ hai vật ngay khi va chạm mềm là một hệ kín nên động lượng của hệ được bảo toàn:
		
		$$m\vec v = (m+ 3m)\vec V.$$
		
		$$ \Rightarrow V = \dfrac{mv}{m + 3m} = \dfrac{v}{4}.$$
	}
	\item \mkstar{3}
	
	
	{Một vật khối lượng $m$ đang chuyển động theo phương ngang với vận tốc $v$ thì va chạm vào vật khối lượng $2m$ đang đứng yên. Sau va chạm, hai vật dính vào nhau và chuyển động với cùng vận tốc. Bỏ qua ma sát, vận tốc của hệ sau va chạm là 
		\begin{mcq}(4)
			\item $\dfrac{v}{3}.$
			\item $v.$
			\item $3v.$
			\item $\dfrac{v}{2}.$
		\end{mcq}
	}
	
	\hideall
	{	
		\textbf{Đáp án: A.}
		
		Hệ hai vật ngay khi va chạm mềm là một hệ kín nên động lượng của hệ được bảo toàn:
		
		$$m\vec v = (m+ 2m)\vec V.$$
		
		$$ \Rightarrow V = \dfrac{mv}{m + 2m} = \dfrac{v}{3}.$$
	}
	\item \mkstar{3}
	
	
	{Một vật khối lượng $\SI{0,7}{kg}$ đang chuyển động theo phương ngang với tốc độ $\SI{5}{m/s}$ thì va vào bức tường thẳng đứng. Nó nảy ngược trở lại với tốc độ $\SI{2}{m/s}$. Chọn chiều dương là chiều bóng nảy ra. Độ thay đổi động lượng của nó là
		\begin{mcq}(4)
			\item $\SI{3,5}{kg\cdot m/s}.$
			\item $\SI{2,45}{kg\cdot m/s}.$ 
			\item $\SI{4,9}{kg\cdot m/s}.$
			\item $\SI{1,1}{kg\cdot m/s}.$ 
		\end{mcq}
	}
	
	\hideall
	{	
		\textbf{Đáp án: C.}
		
		Độ biến thiên động lượng:
		
		$$\Delta p = \vec p_2 - \vec p_1 = mv' + mv = \SI{4,9}{kg \cdot m/s}.$$
		
		vì chiều dương theo chiều bóng nảy.
	}
	\item \mkstar{3}
	
	
	{Một vật có khối lượng $\SI{2}{kg}$ rơi tự do xuống đất trong khoảng thời gian $\SI{0,5}{s}$. Độ biến thiên động lượng của vật trong khoảng thời gian đó là bao nhiêu? Cho $g = \SI{10}{m/s}^2$. 
		\begin{mcq}(4)
			\item $\SI{5,0}{kg\cdot m/s}.$ 
			\item $\SI{4,9}{kg\cdot m/s}.$  
			\item $\SI{10}{kg\cdot m/s}.$ 
			\item $\SI{0,5}{kg\cdot m/s}.$  
		\end{mcq}
	}
	
	\hideall
	{	
		\textbf{Đáp án: C.}
		
		Ta có:
		
		$$\Delta p = p_2 - p_1 = m(v_2 - v_1).$$
		
		Mà:
		
		$$v_1 = 0; v_2 = gt = \SI{5}{m/s}.$$
		
		$$\Rightarrow \Delta P = mv_2 = \SI{10}{kg \cdot m/s}.$$
	}
	\item \mkstar{2}
	
	
	{Chọn câu phát biểu đúng: Một vật nhỏ $m =\SI{200}{g}$ rơi tự do. Lấy $g = \SI{10}{m/s}^2$. Độ biến thiên động lượng của vật từ thời điểm thứ hai đến thời điểm thứ sáu kể từ lúc bắt đầu rơi là
		\begin{mcq}(4)
			\item $\SI{0,8}{kg\cdot m/s}.$ 
			\item $\SI{8}{kg\cdot m/s}.$ 
			\item $\SI{80}{kg\cdot m/s}.$  
			\item $\SI{800}{kg\cdot m/s}.$ 
		\end{mcq}
	}
	
	\hideall
	{	
		\textbf{Đáp án: B.}
		
		Vật rơi tự do, phương trình vận tốc của vật: $v=gt$.
		
		+ Vận tốc của vật tại thời điểm thứ 2: $v_2 = 2g.$
		
		+ Vận tốc của vật tại thời điểm thứ 6: $v_6 = 6g.$
		
		Biến thiên động lượng lượng của vật từ thời điểm thứ 2 đến thời điểm thứ 6:
		
		$$\Delta p = p_6 - p_2 = mv_6 - mv_2 = 6mg - 2mg = \SI{8}{kg \cdot m/s}.$$
		
	}
	\item \mkstar{2}
	
	
	{Một vật nhỏ khối lượng $m =\SI{2}{kg}$ trượt xuống một đường dốc thẳng nhẳn tại một thời điểm xác định có vận tốc $\SI{3}{m/s}$, sau đó $\SI{4}{s}$ có vận tốc $\SI{7}{m/s}$, tiếp ngay sau đó $\SI{3}{s}$ vật có động lượng là 
		\begin{mcq}(4)
			\item $\SI{6}{kg\cdot m/s}.$ 
			\item $\SI{10}{kg\cdot m/s}.$ 
			\item $\SI{20}{kg\cdot m/s}.$  
			\item $\SI{28}{kg\cdot m/s}.$ 
		\end{mcq}
	}
	
	\hideall
	{	
		\textbf{Đáp án: C.}
		
		Ta có: $$a = \dfrac{v - v_0}{4} = \SI{1}{m/s}^2.$$
		
		Vận tốc sau $\SI{3}{s}$ là:
		
		$$v = v_0 + at = \SI{10}{m/s}.$$
		
		Động lượng của vật
		
		$$ p =mv = \SI{20}{kg \cdot m/s}.$$
	}
	\item \mkstar{2}
	
	
	{Một người có khối lượng $m_1=\SI{50}{kg}$ nhảy từ một chiếc xe có khối lượng $m_2 = \SI{80}{kg}$ đang chuyển động theo phương ngang với vận tốc $v = \SI{3}{m/s}$. Biết vận tốc nhảy của người đối với xe lúc chưa thay đổi vận tốc là $v_0 = \SI{4}{m/s}$. Vận tốc của xe sau khi người ấy nhảy ngược chiều đối với xe là
		\begin{mcq}(4)
			\item $\SI{5,5}{m/s}.$
			\item $\SI{4,5}{m/s}.$ 
			\item $\SI{0,5}{m/s}.$
			\item $\SI{1}{m/s}.$
		\end{mcq}
	}
	
	\hideall
	{	
		\textbf{Đáp án: A.}
		
		Khi người nhảy ngược chiều xe, ta có vận tốc ngược chiều nhau.
		
		$$\Rightarrow v' = v - v_0 = - \SI{1}{m/s}.$$
		
		Áp dụng định luật bảo toàn động lượng cho hệ gồm người và xe ban đầu và khi người nhảy là:
		
		$$\vec p = \vec p_1 + \vec p_2 \Rightarrow p = p_1 + p_2.$$
		
		$$\Leftrightarrow (m_1+m_2)v = m_1v'+m_2V \Rightarrow V = \SI{5,5}{m/s}.$$
		
		
	}
	\item \mkstar{2}
	
	
	{Tên lửa khối lượng $\SI{500}{kg}$ đang chuyển động với vận tốc  $\SI{200}{m/s}$ thì tách ra làm hai phần. Phần bị tháo rời có khối lượng $\SI{200}{kg}$ sau đó chuyển động ra phía sau với vận tốc $\SI{100}{m/s}$ so với phần còn lại. Vận tốc phần còn lại bằng
		\begin{mcq}(4)
			\item $\SI{240}{m/s}.$
			\item $\SI{266,7}{m/s}.$ 
			\item $\SI{220}{m/s}.$
			\item $\SI{400}{m/s}.$
		\end{mcq}
	}
	
	\hideall
	{	
		\textbf{Đáp án: A.}
		
		Áp dụng định luật bảo toàn động lượng
		
		$$MV = m_1v_1 - m_2v_2 = m_1v_1 - m_2(v_{21}-v_1) \Rightarrow v_1= \dfrac{MV + m_2v_{21}}{m_1+m_2} = \SI{240}{m/s}.$$
	}
	\item \mkstar{2}
	
	
	{Hai viên bi có khối lượng $m_1 = \SI{50}{g}$ và $m_2 = \SI{80}{g}$ đang chuyển động ngược chiều nhau và va chạm nhau. Muốn sau va chạm $m_2$ đứng yên còn $m_1$ chuyển động theo chiều ngược lại với vận tốc như cũ. Cho biết $v_1 = \SI{2}{m/s}$ thì vận tốc của $m_2$ trước va chạm bằng 
		\begin{mcq}(4)
			\item $\SI{1}{m/s}.$
			\item $\SI{2,5}{m/s}.$  
			\item $\SI{3}{m/s}.$ 
			\item $\SI{2}{m/s}.$ 
		\end{mcq}
	}
	
	\hideall
	{	
		\textbf{Đáp án: B.}
		
		Áp dụng định luật bảo toàn định lượng lúc trước và sau va chạm:
		
		$$ \vec p_1 = \vec p_2 \Leftrightarrow m_1 \vec v_1 + m_2 \vec v_2 = m_1 \vec v'_1.$$
		
		Chọn chiều dương là chiều chuyển động của xe 1 trước khi va chạm. Ta có:
		
		$$m_1 v_1 - m_2v_2 = - m_1v_1 \Leftrightarrow m_2v_2 = 2m_1v_1 \Rightarrow v_2 = \SI{2,5}{m/s}^2.$$
	}
	\item \mkstar{3}
	
	
	{Tên lửa khối lượng $M=10\ \text{tấn}$ đang chuyển động với vận tốc $v_0=200\ \text{m/s}$ so với Trái Đất thì phụt về phía sau một lượng khí có khối lượng $m=2\ \text{tấn}$ với vận tốc $v_1=500\ \text{m/s}$ so với tên lửa. Xác định vận tốc $v$ của tên lửa sau khi khí phụt ra phía sau, giả sử rằng toàn bộ lượng khí được phụt ra cùng lúc. 
		\begin{mcq}(4)
			\item $\SI{325}{\meter/\second}.$
			\item  $\SI{35}{\meter/\second}.$
			\item $\SI{25}{\meter/\second}.$
			\item $\SI{32}{\meter/\second}.$
		\end{mcq}
	}
	
	\hideall
	{	
		\textbf{Đáp án: A.}
		
		Chọn chiều dương là chiều chuyển động của tên lửa, gọi $v'$ là vận tốc khí so với Trái đất. Áp dụng định luật bảo toàn động lượng: 
		\begin{equation*}
			\vec{p}_1=\vec{p}_2  
			\Leftrightarrow M\vec{v_0}=(M-m)\vec{v} + m\vec{v'}.
		\end{equation*}
		Chiếu biểu thức lên chiều dương 
		\begin{equation*}
			Mv_0=(M-m)v+m(-v_1+v_0)
			\Rightarrow v= \frac{Mv_0 - m(-v_1+v_0)}{M-m}=\SI{325}{\meter/\second}.
		\end{equation*}
	
	}
		\item \mkstar{3}
	
	
	{Một xe ô tô có khối lượng $m_1 = 6\ \text{tấn}$ chuyển động thẳng với vận tốc $v_1=3\ \text{m/s}$, đến tông và dính vào một xe gắn máy đang đứng yên có khối lượng $m_2 = 200\ \text{kg}$. Tính vận tốc của các xe sau va chạm.
		\begin{mcq}(4)
			\item $\SI{1.9}{\meter/s}.$
			\item $\SI{3.9}{\meter/s}.$ 
			\item $\SI{4.9}{\meter/s}.$
			\item $\SI{2.9}{\meter/s}.$
		\end{mcq}
	}
	
	\hideall
	{	
		\textbf{Đáp án: D.}
		
		Xem hệ hai xe là hệ cô lập, áp dụng định luật bảo toàn động lượng của hệ: 
		$$m_1 \cdot \vec{v_1} = (m_1+m_2)\cdot \vec{v}.$$
		Khi xe ô tô tông và dính vào xe máu nên sẽ chuyển động theo chiều cũ. Vận tốc của các xe là:
		$$v=\frac{m_1 \cdot v_1}{m_1+m_2}= \SI{2.9}{\meter/s}.$$
	}
\end{enumerate}

\hideall
{
	\begin{center}
		\textbf{BẢNG ĐÁP ÁN}
	\end{center}
	\begin{center}
		\begin{tabular}{|m{2.8em}|m{2.8em}|m{2.8em}|m{2.8em}|m{2.8em}|m{2.8em}|m{2.8em}|m{2.8em}|m{2.8em}|m{2.8em}|}
			\hline
			1.B  & 2.B  & 3.C  & 4.A  & 5.D  & 6.A  & 7.A  & 8.A  & 9.B  & 10.B  \\
			\hline
			11.C  & 12. B  & 13.B  & 14.C  & 15.B  & 16.A  & 17.C  & 18.C  & 19.B  & 20.C  \\
			\hline
			21.A  & 22.A  & 23.B  & 24.A  & 25.D  &   &   &   &   &  \\
			\hline
		\end{tabular}
	\end{center}
}