
\setcounter{section}{0}

\begin{enumerate}[label=\bfseries Câu \arabic*:]
	\item \mkstar{2}
	
	{
		Em hãy kể tên một số dạng nhiên liệu hóa thạch và giải thích tại sao nguồn nhiên liệu này lại là nguồn năng lượng không tái tạo?
	}
	
	\hideall{
		
		Nhiên liệu hóa thạch: than đá, dầu mỏ, hạt nhân,..
		
		Do không thể được bổ sung, làm lại trong một thời gian ngắn được coi là năng lượng không tái tạo. 
	}
	
	\item \mkstar{2}
	
	
	{
		Có những loại năng lượng tái tạo nào?
	}
	
	\hideall
	{
		Nguồn năng lượng tái tạo: nguồn năng lượng sinh học, bức xạ mặt trời, gió, nước chảy, địa nhiệt, nhiệt đại dương, Hydrogen.
		
	}
	\item \mkstar{2}
	
	
	{
		Làm thế nào thu được năng lượng tái tạo?
	}
	
	\hideall
	{
		- Thủy điện: xây dựng nhà máy thủ điện để khai thác động năng của dòng nước để tạo ra điện. Máy phát điện có khả năng chuyển cơ năng thành điện năng nhờ hiện tượng cảm ứng điện từ.
		
		- Công nghệ nhiên liệu sinh học: Năng lượng sinh học là năng lượng bắt nguồn từ quá trình chuyển đổi sinh khối, trong đó sinh khối có thể được sử dụng trực tiếp như nhiên liệu hoặc được xử lí thành các chất lỏng và chất khí.
		
		- Công nghệ thu năng lượng mặt trời: Năng lượng mặt trời là quá trình chuyển đổi năng lượng từ ánh sáng mặt trời thành các dạng năng lượng có thể sử dụng. Bếp năng lượng mặt trời là một thiết bị dùng gương cầu lõm hay thấu kính để hội tụ ánh nắng vào điểm cần đun nấu, máy nước nóng năng lượng mặt trời dựa vào nguyên lí đối lưu nhiệt tự nhiên và hiệu ứng lồng kính giúp biến đổi quang năng thành nhiệt năng.
		
		- Công nghệ thu năng lượng gió: Năng lượng gió là động năng của gió được khai thác để sản xuất điện thông qua tuabin gió. Các tuabin gió hoạt động sẽ có tác dụng chuyển năng lượng của gió (cơ năng) thành điện năng.
		
		- Năng lượng đại dương: 
		
		+ Năng lượng thủy triều: năng lượng tiềm năng liên quan tới triều cường có được khai thác bằng cách xây dựng đập hoặc công trình xây dựng khác ngang qua của sông để tạo thành dòng nước có thể làm quay tuabin máy phát điện. 
		
		+ Năng lượng sóng: động năng và thế năng của sóng đại dương có thể được khai thác để sản xuất điện.
		
		+ Năng lượng nhiệt đại dương: nhiệt độ giữa bề mặt nước biển và nước sâu có sự chênh lệch, có thể được khai thác để chuyển đổi năng lượng nhiệt đại dương thành điện năng.
		
		- Công nghệ thu năng lượng địa nhiệt: Năng lượng địa nhiệt từ các nguồn đá nóng, nước nóng ngầm dưới đất như núi lửa, suối nước nóng...
	}
	\item \mkstar{3}
	
	
	{
		Kể tên ba nhà máy thủy điện có công suất lớn ở Việt Nam. Nhận xét lợi ích của nhà máy thủy điện mang lại và nguy cơ gây mất cân bằng hệ sinh thái do nhà máy thủy điện có thể gây ra. Tại sao các nhà máy thủy điện chủ yếu được xây dựng ở miền núi?
	}
	
	\hideall
	{
		Ba nhà máy thủy điện công suất lớn: 
		
		- Nhà máy thủy điện Sơn La.
		
		- Nhà máy thủy điện Hòa Bình.
		
		- Nhà máy thủy điện Lai Châu.
		
		Nhận xét: 
		
		- Ưu điểm:
		
		+ Thúc đẩy các khả năng kinh tế.
		
		+ Bảo tồn các hệ sinh thái.
		
		+ Linh hoạt.
		
		+ Linh hoạt.
		
		+ Tương đối sạch.
		
		+ Góp phần vào phát triển bền vững.
		
		+ Giảm phát thải.
		
		+ Sử dụng nước đa mục tiêu.
		
		+ Vai trò năng lượng của thủy điện.
		
		+ Góp phần phát triển cơ sở hạ tầng.
		
		+ Cải thiện công bằng xã hội.
		
		+ Kinh tế dự án thuỷ điện.
		
		- Nhược điểm: 
		
		+ Nhấn chìm rừng đầu nguồn.
		
		+ Dòng chảy cạn kiệt.
		
		+ Thay đổi dòng chảy.
		
		+ Ngăn dòng trầm tích.
		
		+ Hạn chế cấp nước cho các mục tiêu khác.
		
		+ Thay đổi xấu chất lượng nước.
		
		+ Một trong những nguyên nhân gây lũ lụt.
		
		Nhà máy thủy điện là những nhà máy được vận hành dựa trên năng lượng nước. Đa số năng lượng thủy điện có được từ thế năng của nước được tích tại các đập nước làm quay một tuốc bin nước và máy phát điện. Vì thế với địa hình cao, bị cắt xẻ mạnh cùng với nhiều sông lớn là điều kiện thuận lợi để xây dựng các đập chứa nước. Vì thế, ở vùng Trung du và miền núi Bắc Bộ có nhiều nhà máy thủy điện lớn như Hòa Bình, Sơn La,....
	}
	\item \mkstar{3}
	
	
	{
		Chu trình sản xuất khí sinh học như thế nào? Các yêu cầu thiết kế công trình khí sinh học nhỏ như thế nào? Tại sao cần thực hiện các yêu cầu đó?
	}
	
	\hideall
	{
		Các chất thải hữu cơ, chất thải chăn nuôi trở thành nguyên liệu được đưa vào hầm biogas kín khí. Các chất này phân hủy và lên men dưới tác động của vi sinh vật. Một phần nhỏ các nguyên tố khác như Nitơ, phốt pho,.. được thất thoát qua quá trình phân hủy từ hầm biogas. Khí sinh ra được đưa vào hệ thống lọc tạp chấ sau đó chuyển lên nhà máy phát điện.
		
		Yêu cầu cơ bản để đảm bảo an toàn cần có hệ thống xả khí để tránh nổ hầm.
		
	}
	\item \mkstar{2}
	
	{
		Nhận xét lợi ích các nhà máy này điện mặt trời mang lại. Giải thích tại sao phần lớn các dự án điện Mặt Trời ở Việt Nam tập trung nhiều ở Miền Trung và Miền Nam.
	}
	
	\hideall{
		
		Ưu điểm của điện mặt trời là nguồn năng lượng tái tạo, vĩnh cửu, không lo cạn kiệt. Nguồn điện từ mặt trời giúp tiết kiệm chi phí tiền điện do không phải sử dụng điện lưới; không chi phí vận hành, chi phí bảo trì thấp. Đồng thời, thân thiện với môi trường vì trong quá trình vận hành không gây ra tiếng ồn, khói bụi..
		
		Với đặc điểm khí hậu phù hợp, các dự án điện mặt trời hiện nay đang chủ yếu tập trung tại khu vực miền Nam và Nam Trung Bộ. Đây là khu vực có tỷ trọng phụ tải chiếm khoảng $50\%$ so với toàn quốc. Vì vậy, với việc đưa vào vận hành các dự án năng lượng tái tạo, phần nào sẽ giảm bớt sự thiếu hụt về năng lượng tại miền Nam. Qua đó, tăng cường an ninh cung ứng điện và giảm căng thẳng trong vận hành hệ thống điện.
		
		
	}
	
	\item \mkstar{2}
	
	
	{
		Tìm hiểu một số mô hình cánh quạt tuabin gió và nhận xét ưu điểm, nhược điểm của tuabin gió trục đứng và tuabin gió trục ngang.
	}
	
	\hideall
	{
		Tóm lại, các loại tuabin gió chính hiện có:
		
		- Tua bin gió trục ngang (HAWT).
		
		- Tua bin gió trục thẳng đứng (VAWT).
		
		- Tua bin gió xuôi.
		
		- Tua bin gió không cánh.
		
		- Bộ cánh quạt quay ngược (Counter-Rotating).
		
		\textbf{Tua bin gió trục ngang:} Đây là loại tuabin phổ biến nhất và có thể dễ dàng nhận ra bởi thiết kế giống như cánh quạt của chúng. HAWT tạo ra điện bằng cách sử dụng lực nâng với các cánh có hình dạng giống như những chiếc máy bay, giống như cách một chiếc máy bay cất cánh trên mặt đất. Mặc dù, các cánh của tuabin gió có tỷ lệ lực nâng cao hơn nhiều lần so với cánh của một máy bay thương mại. Cho đến nay, đây là loại tuabin gió hiệu quả nhất và được nghiên cứu kỹ lưỡng.
		
		HAWT được thiết lập trên cao trên các tháp lớn để tận dụng tốc độ gió ở độ cao lớn. Mặc dù điều này mang lại hiệu quả tốt hơn nhưng lại gây khó khăn cho việc sửa chữa. Hộp nan hoa, hoặc hộp nhỏ phía sau rôto, chứa hộp số, máy phát điện, phanh tốc độ, cơ cấu yaw và máy đo gió. Hộp số được dùng để khuếch đại tốc độ tuabin, cho phép máy phát điện tạo ra.
		
		HAWT phải giữ vuông góc với gió để tối đa hóa hiệu quả. Cơ chế yaw quay các tuabin theo hướng gió để đảm bảo tuabin vẫn thẳng hàng với hướng gió. HAWT cũng không hoạt động tốt ở tốc độ gió cực mạnh. Khi máy đo gió nhận được tốc độ gió có thể làm hỏng tuabin, hệ thống hãm tốc độ sẽ được kích hoạt để ngăn tuabin quay quá nhanh. Hãy xem bài viết sau đây để có cái nhìn chuyên sâu hơn: cách thức hoạt động của tuabin gió.
		
		\textbf{Tua bin gió trục thẳng đứng}: Chúng có xu hướng ít phổ biến hơn ở quy mô công nghiệp. VAWT thường được sử dụng ở quy mô nhỏ hơn HAWT và chủ yếu được sử dụng trên các mái nhà như một nguồn năng lượng bổ sung. Chúng thường ở gần mặt đất, làm cho chúng kém hiệu quả hơn do tốc độ gió thấp hơn nhưng dễ bảo trì và bảo dưỡng hơn.
		
		Không phải tất cả các VAWT đều giống nhau. Chúng có hai thiết kế chính, Darrieus và Savonius. Cả hai đều rất khác nhau về cách chúng thu nhận năng lượng gió và mỗi loại đều có một số thiết kế phụ trợ cho các mục đích khác nhau.
		
		
		
	}
	\item \mkstar{3}
	
	
	{
		Tại sao phải khai thác nguồn năng lượng mới?
	}
	
	\hideall
	{
		Chuyển dịch năng lượng (Energy Transition) là sự thay đổi chính sách, cơ cấu, công nghệ của ngành năng lượng, từ sản xuất, tiêu thụ các nguồn năng lượng hóa thạch truyền thống như than, dầu, khí tự nhiên sang các nguồn năng lượng tái tạo, bền vững như gió, mặt trời, sinh khối...
		
		Trong khi nhiên liệu hóa thạch là loại nhiên liệu phải mất hàng trăm triệu năm để hình thành ở các dạng khác nhau như than đá, dầu mỏ, khí đốt... tùy vào điều kiện môi trường, thì tốc độ tiêu thụ của con người quá nhanh. Điều này đã đặt ra sức ép lớn trong việc bảo đảm nhu cầu năng lượng cũng như an ninh năng lượng của mỗi quốc gia. Bởi vậy, việc hướng tới sử dụng các nguồn năng lượng tái tạo là xu thế tất yếu, một trong những cách giúp giải quyết vấn đề tăng nhu cầu năng lượng hiện nay.
	}
	\item \mkstar{3}
	
	
	{
		Liệt kê một số khí thải độc hại được sinh ra trong quá trình đốt nhiên liệu hóa thạch.
	}
	
	\hideall
	{
		Khi đốt nhiên liệu hóa thạch, chúng thải ra nhiều chất thải độc hại như benzen, formaldehyde.
		
		Quá trình trên cũng thải ra khoảng 21,3 tỷ tấn CO$_2$ mỗi năm, tăng lượng khí nhà kính, tạo ra nhiều chất ô nhiễm khác như NO$_2$, SO$_2$, hợp chất hữu cơ dễ bay hơi và kim loại nặng.
	}
	\item \mkstar{3}
	
	
	{
		Việc xây dựng và vận hành các nhà máy điện gió có tác động tiêu cực gì?
	}
	
	\hideall
	{
		Tác động tiêu cực:
		
		- Tác động tới môi trường.
		
		- Tác động tới cảnh quan.
		
		- Tác động tới cảnh quan.
		
		- Tác động đến hệ sinh thái biển.
		
		- Tác động đến sức khỏe con người.
		
		- Tác động đến sinh vật.
		
		- Rung động tần số thấp.
		
		- Ảnh hưởng đến sóng vô tuyến.
		
		- Ảnh hưởng đến đường hàng không.
		
		- Ảnh hưởng của tiếng ồn.
	}
\end{enumerate}