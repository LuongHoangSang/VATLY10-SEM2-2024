\setcounter{section}{0}
\section{Trắc nghiệm}
\begin{enumerate}[label=\bfseries Câu \arabic*:]
	\item \mkstar{1}
	
	
	{Trong một mạch kín gồm nguồn điện có suất điện động $\calE$, điện trở trong $r$ và mạch ngoài có điện trở $R$. Hệ thức nêu lên mối quan hệ giữa các đại lượng trên với cường độ dòng điện $I$ chạy trong mạch là
		\begin{mcq}(2)
			\item $H=\dfrac{R_\text{N}}{r}$. 
			\item $H=\dfrac{R_\text{N}}{R_\text{N}+r}$. 
			\item $H=\dfrac{R_\text{N}+r}{r}$. 
			\item $H=\dfrac{r}{R_\text{N}}$. 
		\end{mcq}
	}
	
	\hideall
	{	\textbf{Đáp án: B} 
		
		Trong một mạch điện kín, nếu mạch ngoài chỉ gồm điện trở $R_\text{N}$ thì hiệu suất của nguồn điện có điện trở trong $r$: $H=\dfrac{R_\text{N}}{R_\text{N}+r}$.
		
	}
	\item \mkstar{2}
	
	
	{	Định luật Ôm đối với toàn mạch được phát biểu rằng: 
		\begin{mcq}
			\item Cường độ dòng điện chạy trong mạch điện kín tỉ lệ thuận với suất điện động của nguồn điện và tỉ lệ nghịch với điện trở toàn phần của mạch đó.
			\item Cường độ dòng điện chạy trong mạch điện kín tỉ lệ nghịch với suất điện động của nguồn điện và tỉ lệ thuận với điện trở toàn phần của mạch đó. 
			\item Cường độ dòng điện chạy trong mạch điện kín tỉ lệ thuận với suất điện động của nguồn điện và  điện trở toàn phần của mạch đó.
			\item 	Cường độ dòng điện chạy trong mạch điện kín tỉ lệ nghịch với suất điện động của nguồn điện và điện trở toàn phần của mạch đó.
		\end{mcq}
	}
	
	\hideall
	{	\textbf{Đáp án: A} 
		
		Định luật Ôm đối với toàn mạch được phát biểu rằng:	Cường độ dòng điện chạy trong mạch điện kín tỉ lệ thuận với suất điện động của nguồn điện và tỉ lệ nghịch với điện trở toàn phần của mạch đó.	
		
	}
	\item \mkstar{2}
	
	
	{	Mạch điện gồm điện trở $R = 2\, \Omega$ mắc thành mạch điện kín với nguồn $\calE= 3\, \text{V},\, r = 1 \, \Omega$  thì cường độ dòng điện trong mạch bằng
		\begin{mcq}(4)
			\item 1 A. 
			\item  2 A.
			\item  3 A. 
			\item  4 A. 
		\end{mcq}
	}
	
	\hideall
	{	\textbf{Đáp án: A} 
		
		$I= \dfrac{\calE}{R+r}$= $\dfrac{3}{2+1}=1 \text{A}$. 
		
	}
	\item \mkstar{2}
	
	
	{ Người ta mắc hai cực của một nguồn điện $(\calE,r)$ với một biến trở. Thay đổi điện trở của biến trở, đo hiệu điện thế $U$ giữa hai cực của nguồn điện và cường độ dòng điện $I$ chạy trong mạch. Biết khi $I$ = 0 thì $U= \text{4,5}\, \text{V}$ và khi $I = \text{2,0}\, \text{A}$ thì $U = \text{4,0} \text{V}$. Tính $\calE$, $r$.
		\begin{mcq}(2)
			\item $\calE= \text{4,5}\,\text{V},\, r=\, \text{0,5} \Omega$. 
			\item  $\calE= \text{4,5}\,\text{V},\, r=\, \text{0,25} \Omega$.
			\item  $\calE= \text{4,5}\,\text{V},\, r=\, \text{0,2} \Omega$.
			\item  $\calE= \text{4,0}\,\text{V},\, r=\, \text{0,25} \Omega$.
		\end{mcq}
	}
	
	\hideall
	{	 \textbf{Đáp án: B} 
		
		Ta có: $U = \calE- I\cdot r$.
		
		Với $I=0$ thì  $U=\calE= \text{4,5}\, \text{V}$.
		
		Với $I = \text{2,0}\ \text{A}$ thì $U=\calE- I\cdot r$ $\Leftrightarrow$ r= \text{0,25}\, $\Omega$. 
		
		
		
	}
	\item \mkstar{2}
	
	
	{	Cho một mạch điện có điện trở không đổi. Khi dòng điện trong mạch là 2 A thì công suất tiêu thụ của mạch điện là 100 W. Khi dòng điện trong mạch là 1 A thì công suất tiêu thụ của mạch là
		\begin{mcq}(4)
			\item 25 W. 	
			\item 200 W.
			\item 400 W.
			\item 50 W.  
		\end{mcq}
	}
	
	\hideall
	{			
		\textbf{Đáp án: A} 
		
		$\calP= I^2\cdot R$. Mà $R$ không đổi nên $\dfrac{\calP_1}{\calP_2}$= $\dfrac{I_1^2}{I_2^2}$. 
		
		Từ đó, ta tìm được $\calP_2=25\, \text{W}.$
		
	}

		\item \mkstar{1}
	
	
	{Trong một mạch kín gồm nguồn điện có suất điện động $\calE$, điện trở trong $r$ và mạch ngoài có điện trở $R$. Hệ thức nêu lên mối quan hệ giữa các đại lượng trên với cường độ dòng điện $I$ chạy trong mạch là
		\begin{mcq}(4)
			\item $I =\dfrac{ \calE}{R}$. 
			\item $I = \calE \sqrt{\dfrac{\calE}{R}}$. 
			\item $I = \dfrac{\calE}{R+r}$. 
			\item $I = \dfrac{\calE}{r}$. 
		\end{mcq}
	}
	
	\hideall
	{	\textbf{Đáp án: C} 
		
		
		
	}
	\item \mkstar{1}
	
	
	{Đối với mạch điện kín gồm nguồn điện với mạch ngoài là điện trở thì hiệu điện thế mạch ngoài
		\begin{mcq}
			\item giảm khi cường độ dòng điện trong mạch tăng.
			\item tỉ lệ thuận với cường độ dòng điện chạy trong mạch.
			\item tăng khi cường độ dòng điện trong mạch tăng.
			\item tỉ lệ nghịch với cường độ dòng điện chạy trong mạch.
		\end{mcq}
	}
	
	\hideall
	{		\textbf{Đáp án: A.}
		
	
		
	}
	\item \mkstar{1}
	
	
	{Biểu thức nào sau đây là \textbf{không} đúng?
		\begin{mcq}(4)
			\item $I = \dfrac{\calE}{R+r}$.
			\item $I = \dfrac{U}{R}$.
			\item $E = U - Ir$.
			\item $E = U + Ir$
		\end{mcq}
	}
	
	\hideall
	{		\textbf{Đáp án: C.}
		
		
		
	}
	\item \mkstar{2}
	
	
	{Cho một mạch điện gồm một pin $\SI{1,5}{V}$ có điện trở trong là $\SI{0,5}{\Omega}$ nối với mạch ngoài là một điện trở $\SI{2,5}{\Omega}$. Cường độ dòng điện trong toàn mạch là
		\begin{mcq}(4)
			\item $I = \SI{3}{A}$.
			\item $I = \SI{0,6}{A}$.
			\item $I = \SI{0,5}{A}$.
			\item $I = \SI{2}{A}$.
		\end{mcq}
	}
	
	\hideall
	{		\textbf{Đáp án: C.}
		
		Ta có: 
		
		$$I = \dfrac{\calE}{R+r} = \SI{0,5}{A}.$$ 
		
	}
	\item \mkstar{2}
	
	
	{Một mạch điện có nguồn là một pin $\SI{9}{V}$, điện trở trong $\SI{0,5}{\Omega}$ và mạch ngoài gồm hai điện trở $\SI{8}{\Omega}$ mắc song song. Cường độ dòng điện toàn mạch là
		\begin{mcq}(4)
			\item $\SI{2}{A}$.
			\item $\SI{4,5}{A}$.
			\item $\SI{1}{A}$.
			\item $\SI{3}{A}$.
		\end{mcq}
	}
	
	\hideall
	{		\textbf{Đáp án: A.}
		
		Khi hai điện trở mắc song song
		
		$$R_\text{tđ} = \dfrac{R}{2} = \SI{4}{\Omega}.$$
		
		Cường độ dòng điện trong mạch là:
		
		$$I = \dfrac{\calE}{R+r} = \SI{2}{A}.$$ 
		
	}
	\item \mkstar{1}
	
	
	{Trong một mạch kín mà điện trở ngoài là $\SI{10}{\Omega}$, điện trở trong là $\SI{1}{\Omega}$ có dòng điện đi qua là $\SI{2}{A}$. Hiệu điện thế hai đầu nguồn và suất điện động của nguồn là
		\begin{mcq}(2)
			\item $\SI{10}{V}$ và $\SI{12}{V}$.    
			\item $\SI{20}{V}$ và $\SI{22}{V}$.
			\item $\SI{10}{V}$ và $\SI{2}{V}$.    
			\item $\SI{2,5}{V}$ và $\SI{0,5}{V}$.
		\end{mcq}
	}
	
	\hideall
	{		\textbf{Đáp án: B.}
		
		Ta có: $$U = IR_\text{N} = \SI{20}{V}$$
		$$E = (R_\text{N} + r)I = \SI{22}{V}.$$
		
	}
	\item \mkstar{2}
	
	
	{Một mạch điện có điện trở ngoài bằng 5 lần điện trở trong. Khi xảy ra hiện tượng đoản mạch thì tỉ số giữa cường độ dòng điện đoản mạch và cường độ dòng điện không đoản mạch là
		\begin{mcq}(2)
			\item 5.
			\item 6.
			\item 4.
			\item Không xác định được.
		\end{mcq}
	}
	
	\hideall
	{		\textbf{Đáp án: B.}
		
		Ta có:
		
		$$I = \dfrac{\calE}{R+r} = \dfrac{\calE}{5r+r} = \dfrac{\calE}{6r}.$$
		
		Khi xảy ra đoản mạch:
		
		$$R_\text{N} = 0 \Rightarrow I' = \dfrac{\calE}{r} \Rightarrow I' =6I.$$
		
		
		
	}
	
	\item \mkstar{1}
	
	
	{Cho một đoạn mạch gồm hai điện trở $R_1$ và $R_2$ mắc song song và mắc vào một hiệu điện thế không đổi. Nếu giảm trị số của điện trở $R_2$ thì
		\begin{mcq}(2)
			\item độ sụt thế trên $R_2$ giảm.  
			\item dòng điện qua $R_1$ không thay đổi.
			\item dòng điện qua $R_1$ tăng lên.  
			\item công suất tiêu thụ trên $R_2$ giảm.
		\end{mcq}
	}
	
	\hideall
	{		\textbf{Đáp án: B.}
		
		
		
	}
	\item \mkstar{1}
	
	
	{Hiện tượng đoản mạch xảy ra khi
		\begin{mcq}
			\item nối hai cực của một nguồn điện bằng dây dẫn có điện trở rất nhỏ.
			\item sử dụng các dây dẫn ngắn để mắc mạch điện.
			\item không mắc cầu chì cho mạch điện kín.
			\item dùng pin (hay ác quy) để mắc một mạch điện kín.
		\end{mcq}
	}
	
	\hideall
	{		\textbf{Đáp án: A.}
		
		Hiện tượng đoản mạch xảy ra khi nối hai cực của một nguồn điện bằng dây dẫn có điện trở rất nhỏ.
		
	}
	\item \mkstar{1}
	
	
	{Đối với mạch điện kín gồm nguồn điện với mạch ngoài là điện trở thì hiệu điện thế mạch ngoài
		\begin{mcq}
			\item tỉ lệ thuận với cường độ dòng điện chạy trong mạch.
			\item tăng khi cường độ dòng điện trong mạch tăng.
			\item giảm khi cường độ dòng điện trong mạch tăng.
			\item tỉ lệ nghịch với cường độ dòng điện chạy trong mạch.
		\end{mcq}
	}
	
	\hideall
	{		\textbf{Đáp án: C.}
		
		
		
	}
	\item \mkstar{1}
	
	
	{Phát biểu nào sau đây là \textbf{không} đúng?
		\begin{mcq}
			\item Cường độ dòng điện trong đoạn mạch chỉ chứa điện trở $R$ tỉ lệ với hiệu điện thế $U$ giữa hai đầu đoạn mạch và tỉ lệ nghịch với điện trở $R$.
			\item Cường độ dòng điện trong mạch kín tỉ lệ thuận với suất điện động của nguồn điện và tỉ lệ nghịch với điện trở toàn phàn của mạch.
			\item Công suất của dòng điện chạy qua đoạn mạch bằng tích của hiệu điện thế giữa hai đầu đoạn mạch và cường độ dòng điện chạy qua đoạn mạch đó.
			\item Nhiệt lượng toả ra trên một vật dẫn tỉ lệ thuận với điện trở của vật, với cường độ dòng điện và với thời gian dòng điện chạy qua vật.
		\end{mcq}
	}
	
	\hideall
	{		\textbf{Đáp án: D.}
		
		
		
	}
	\item \mkstar{1}
	
	
	{Khi xảy ra hiện tượng đoản mạch, thì cường độ dòng điện trong mạch
		\begin{mcq}(2)
			\item tăng rất lớn.
			\item tăng giảm liên tục.   
			\item giảm về 0.	
			\item không đổi so với trước.
		\end{mcq}
	}
	
	\hideall
	{		\textbf{Đáp án: A.}
		
		Hiện tượng đoản mạch xảy ra khi điện trở mạch ngoài không đáng kể nên cường độ dòng điện chạy trong mạch điện kín đạt giá trị lớn nhất.
		
	}
	\item \mkstar{1}
	
	
	{Nhận xét nào sau đây đúng? Theo định luật Ôm cho toàn mạch thì cường độ dòng điện cho toàn mạch
		\begin{mcq}
			\item tỉ lệ nghịch với suất điện động của nguồn;
			\item tỉ lệ nghịch điện trở trong của nguồn;
			\item tỉ lệ nghịch với điện trở ngoài của nguồn;
			\item tỉ lệ nghịch với tổng điện trở trong và điện trở ngoài. 
		\end{mcq}
	}
	
	\hideall
	{		\textbf{Đáp án: D.}
		
		
		
	}
		\item \mkstar{1}
	
	
	{Hiệu suất của nguồn điện được xác định bằng
		\begin{mcq}
			\item tỉ số giữa công có ích và công toàn phần của dòng điện trên mạch.
			\item tỉ số giữa công toàn phần và công có ích sinh ra ở mạch ngoài.
			\item công của dòng điện ở mạch ngoài.
			\item nhiệt lượng tỏa ra trên toàn mạch.
		\end{mcq}
	}
	
	\hideall
	{		\textbf{Đáp án: A.}
		
		
		
	}
	\item \mkstar{2}
	
	
	{
		Một nguồn điện có điện trở trong $\SI{0.1}{\Omega}$ được mắc với điện trở $\SI{4.8}{\Omega}$ thành mạch kín. Khi đó hiệu điện thế giữa hai cực của nguồn điện là $\SI{12}{\volt}$. Suất điện động của nguồn điện là
		\begin{mcq}(4)
			\item $\calE= \SI{12.00}{\volt}$.
			\item $\calE= \SI{12.25}{\volt}$.
			\item $\calE= \SI{14.50}{\volt}$.
			\item $\calE= \SI{11.75}{\volt}$.
		\end{mcq}
	}
	
	\hideall
	{		\textbf{Đáp án: B.}
		
				Cường độ dòng điện trong mạch:
			$$
			I=\dfrac {U_\text N}{R}=\dfrac{\SI{12}{\volt}}{\SI{4.8}{\Omega}} = \SI{2.5}{\ampere}.
			$$
			
			Suất điện động của nguồn:
			$$
			\calE = U_{\text N}+Ir=\SI{12}{\volt} + \SI{2.5}{\ampere} \cdot \SI{0.1}{\Omega} = \SI{12.25}{\volt}.
			$$
		
	}
\end{enumerate}


\hideall
{
	\begin{center}
		\textbf{BẢNG ĐÁP ÁN}
	\end{center}
	\begin{center}
		\begin{tabular}{|m{2.8em}|m{2.8em}|m{2.8em}|m{2.8em}|m{2.8em}|m{2.8em}|m{2.8em}|m{2.8em}|m{2.8em}|m{2.8em}|}
			\hline
			1.B  & 2.A  & 3.A  & 4.B  & 5.A  & 6.C  & 7.A  & 8.C  & 9.C  & 10.A  \\
			\hline
			11.B  & 12.B  & 13.B  & 14.A  & 15.C  & 16.D  & 17.A  & 18.D  & 19.A  & 20.B  \\
			\hline
			
		\end{tabular}
	\end{center}
}
\section{Tự luận}
\begin{enumerate}[label=\bfseries Câu \arabic*:]
		\item \mkstar{2}
		
		
		{
		Một nguồn điện có suất điện động $E = \SI{6}{V}$, điện trở trong $r = \SI{2}{\Omega}$, mạch ngoài có điện trở $R$. Để công suất tiêu thụ ở mạch ngoài là $\SI{4}{W}$ thì điện trở $R$ phải có giá trị là bao nhiêu?
		}
		
		\hideall
		{	
			
			Công suất tiêu thụ mạch ngoài là $\calP = RI^2$
			
			Cường độ dòng điện trong mạch là
			
			$$I = \dfrac{\calE}{R+r}.$$
			
			Suy ra:
			
			$$\calP = R \left(\dfrac{\calE}{R+r}\right)^2.$$
			
			với $E = \SI{6}{V}, r = \SI{2}{\Omega}, \calP = \SI{4}{W}$ ta tính được $R = \SI{1}{\Omega}$.
			
		}
		\item \mkstar{2}
		
		
		{
		Một mạch điện gồm một pin $\SI{9}{V}$, điện trở mạch ngoài $\SI{4}{\Omega}$, cường độ dòng điện trong toàn mạch là $\SI{2}{A}$. Điện trở trong của nguồn là bao nhiêu?
		}
		
		\hideall
		{	
			Ta có:
			
			$$ I = \dfrac{\calE}{R+r}.$$
			
			Nên:
			
			$$r = \dfrac{\calE}{I} - R = \SI{0,5}{\Omega}.$$
			
			
			
		}
		\item \mkstar{2}
		
		
		{
		Người ta mắc hai cực của nguồn điện với một biến trở có thể thay đổi từ 0 đến vô cực. Khi giá trị của biến trở rất lớn thì hiệu điện thế giữa hai cực của nguồn điện là $\SI{4,5}{V}$. Giảm giá trị của biến trở đến khi cường độ dòng điện trong mạch là $\SI{2}{A}$ thì hiệu điện thế giữa hai cực của nguồn điện là $\SI{4}{V}$. Tính suất điện động và điện trở trong của nguồn điện.
		}
		
		\hideall
		{	
			Khi biến trở rất lớn suy ra $I = 0$.
			
			Khi đó $U_\text{N} = \calE - Ir = \calE = \SI{4,5}{V} $.
			
			Khi $I = \SI{2}{A}, U_N = \SI{4}{V}$.
			
			Ta có:
			
			$$R_\text{N} = \dfrac{U_\text{N}}{I} = \SI{2}{\Omega}.$$
			
			Mặt khác:
			
			$$R_\text{N} +r = \dfrac{\calE}{I} \Rightarrow r = \SI{0,25}{\Omega}.$$
			
		}
		\item \mkstar{3}
		
		
		{
		Hai điện trở $R_1 = 2\ \Omega$, $R_2 = 6\ \Omega$ mắc vào nguồn ($\calE, r$). Khi $R_1$, $R_2$ nối tiếp, cường độ trong mạch $I_\text{N} = \SI{0,5}{A}$. Khi $R_1$, $R_2$ song song, cường độ mạch chính $I_\text{S} = \SI{1,8}{A}$. Tìm $\calE, r$.
		}
		
		\hideall
		{		
			- Khi $R_1$ nối tiếp $R_2$ 
			
			Điện trở trong mạch
			$$R_\text{N} = R_1 +R_2 =8\ \Omega.$$
			
			Cường độ dòng điện trong mạch nối tiếp
			$$I_\text{N} =\dfrac{\calE}{R_\text{N}+r}\Rightarrow \text{0,5} =\dfrac{\calE}{8+r}\Rightarrow 4+\text{0,5}r =\calE\ (1).$$
			
			- Khi $R_1$ song song $R_2$ 
			
			Điện trở trong mạch
			$$R'_\text{N} = \dfrac{R_1R_2}{R_1+R_2} =\text{1,5}\ \Omega.$$
			
			Cường độ dòng điện trong mạch nối tiếp
			$$I_\text{S} =\dfrac{\calE}{R'_\text{N}+r}\Rightarrow \text{1,8} =\dfrac{\calE}{\text{1,5}+r}\Rightarrow \text{2,7}+\text{1,8}r =\calE \ (2).$$	
			
			Từ (1) và (2) suy ra: $r=1\ \Omega; \calE = \SI{4,5}{V}.$
	
			
		}
		\item \mkstar{3}
		
		
		{
			Cho mạch điện trong đó nguồn điện có điện trở trong $r = 1\ \Omega$. Các điện trở của mạch ngoài $R_1 = 6\ \Omega$, $R_2 = 2\ \Omega$, $R_3 = 3\ \Omega$ mắc nối tiếp nhau. Dòng điện chạy trong mạch là $\SI{1}{A}$. Tính suất điện động của nguồn điện và hiệu suất của nguồn điện.
		}
		
		\hideall
		{		
			
			Điện trở tương đương mạch ngoài: 
			
			$$R = R_1 + R_2 + R_3 = 11\ \Omega.$$
			
			Suất điện động của nguồn
			$$\calE  = I(R+r) =\SI{12}{V}.$$
			
			Hiệu điện thế mạch ngoài
			$$U = IR =\SI{11}{V}.$$ 
			
			Hiệu suất của nguồn
			$$H =\dfrac{U}{\calE} = \text{91,67}\%.$$
			
		}
\end{enumerate}